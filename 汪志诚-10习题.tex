\section{习题10}

\newpage
\subsection{10-1}
试从(10.1.10)出发,以 $\Delta p, \Delta S$ 为自变量,证明
$$ W \propto e^{\frac{1}{2kT} \left( \frac{\partial V}{\partial p} \right)_S (\Delta p)^2 - \frac{1}{2kC_p} (\Delta S)^2}, $$
从而证明
$$ \overline{\Delta S \Delta p} = 0, $$
$$ \overline{(\Delta S)^2} = kC_p, $$
$$ \overline{(\Delta p)^2} = -kT \left( \frac{\partial p}{\partial V} \right)_S. $$

\newpage
\subsection{10-2}
利用式 (10.1.12) 求得的 $\overline{(\Delta T)^2}, \overline{(\Delta V)^2}$ 和 $\overline{\Delta T \Delta V}$ 证明:
$$ \overline{\Delta T \Delta S} = kT, $$
$$ \overline{\Delta p \Delta V} = -kT, $$
$$ \overline{\Delta S \Delta V} = kT \left( \frac{\partial V}{\partial T} \right)_p, $$
$$ \overline{\Delta p \Delta T} = \frac{kT^2}{C_v} \left( \frac{\partial p}{\partial T} \right)_v. $$

\newpage
\subsection{10-3}
试证明开系涨落的基本公式
$$ W \propto e^{-\frac{\Delta T \Delta S - \Delta p \Delta V + \Delta \mu \Delta N}{2kT}}, $$
并据此证明,在 $T, V$ 恒定时,有
$$ \overline{(\Delta N)^2} = kT \left( \frac{\partial N}{\partial \mu} \right)_{T,V}, $$
$$ \overline{(\Delta \mu)^2} = kT \left( \frac{\partial \mu}{\partial N} \right)_{T,V}, $$
$$ \overline{\Delta N \Delta \mu} = kT. $$

\newpage
\subsection{10-4}
试证明,对于顺磁介质,有
$$ W \propto e^{-\frac{C_m}{2kT^2} (\Delta T)^2 - \frac{\mu_0}{2kT} \left( \frac{\partial m}{\partial \mathcal{H}} \right)_T (\Delta m)^2}, $$
并据此证明
$$ \overline{\Delta T \Delta m} = 0, $$
$$ \overline{(\Delta T)^2} = \frac{kT^2}{C_m}, $$
$$ \overline{(\Delta m)^2} = \frac{kT}{\mu_0} \left( \frac{\partial m}{\partial \mathcal{H}} \right)_T. $$

\newpage
\subsection{10-5}
试由式(10.2.1)导出式(10.2.9)。

\newpage
\subsection{10-6}
在18℃的温度下,观察半径为 $0.4 \times 10^{-6}$m 的粒子在黏度为 $2.78 \times 10^{-3}$Pa·s 的液体中的布朗运动,测得粒子在时间间隔10s的位移平方的平均值为
$$ \overline{x^2} = 3.3 \times 10^{-12}  \text{m}^2. $$
试根据这些数据求玻耳兹曼常量 $k$ 的值。

\newpage
\subsection{10-7}
电流计带有用细丝悬挂的反射镜。由于反射镜受到气体分子碰撞而施加的力矩不平衡,反射镜不停地进行着无规则的扭摆运动。根据能量均分定理,反射镜转动角度 $\varphi$ 的方均值 $\overline{\varphi^2}$ 满足
$$ \frac{1}{2} A \overline{\varphi^2} = \frac{1}{2} kT. $$
对于很细的石英丝,弹性系数 $A = 10^{-13}$ N·m·rad$^{-2}$,计算300K下的 $\sqrt{\overline{\varphi^2}}$。

\newpage
\subsection{10-8}
三维布朗颗粒在各向同性介质中运动,朗之万方程为
$$ \frac{dp_i}{dt} = -\gamma p_i + F_i(t), \quad i = 1, 2, 3. $$
其涨落力满足
$$ \overline{F_i(t)} = 0, $$
$$ \overline{F_i(t) F_j(t')} = 2m\gamma k T \delta_{ij} \delta (t-t'). $$
试证明,经过时间 $t$ 布朗颗粒位移平方的平均值为
$$ \overline{\left[ \mathbf{x} - \mathbf{x}(0) \right]^2} = \sum_{i} \overline{\left[ x_i - x_i(0) \right]^2} = \frac{6kT}{m\gamma} t. $$

\newpage
\subsection{10-9}
在均匀恒定的外电场 $\mathscr{E}$ 作用下,电荷量为 $q$,质量为 $m$ 的布朗颗粒在流体中运动,运动方程为
$$ m \frac{dv}{dt} = -\alpha v + q \mathscr{E} + F(t), $$
$\alpha$ 是黏性阻力系数,$F(t)$ 是涨落力。达到定常状态时,颗粒的平均速度为
$$ \bar{v} = \frac{q\mathscr{E}}{\alpha}. $$
以 $\mu = \frac{\bar{v}}{\mathscr{E}}$ 表示迁移率,试证明迁移率 $\mu$ 与扩散系数 $D$ 间存在关系
$$ \frac{\mu}{D} = \frac{q}{kT}. $$
上述称为爱因斯坦关系。

\newpage
\subsection{10-10}
考虑布朗颗粒在竖直方向的运动。取 $z$ 轴(向上)沿竖直方向,朗之万方程为
$$ m \frac{dv_z}{dt} = -\alpha v_z - mg + F_z(t) $$
(a) 试证明,达到定常状态后,布朗颗粒的平均速度为
$$ \overline{v_z} = -\frac{mg}{\alpha}. $$
(b) 达到定常状态后,布朗颗粒的流量为零,即
$$ J_z = -D \frac{dn}{dz} + n \overline{v_z} = 0, $$
其中 $n(z)$ 为布朗颗粒的密度。试由此导出达到定常状态后布朗颗粒按高度的分布。

