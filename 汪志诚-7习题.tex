\section{习题7}

\newpage
\subsection{7-1}
试根据公式 $p = - \sum_{l} a_{l} \frac{\partial \varepsilon_{l}}{\partial V}$ 证明,对于非相对论粒子
$$\varepsilon = \frac{p^{2}}{2m} = \frac{1}{2m} \left( \frac{2\pi \hbar}{L} \right)^{2} (n_{x}^{2} + n_{y}^{2} + n_{z}^{2})$$
$$(n_{x}, n_{y}, n_{z} = 0, \pm 1, \pm 2, \cdots),$$
有
$$p = \frac{2}{3} \frac{U}{V}$$
上述结论对于玻耳兹曼分布、玻色分布和费米分布都成立。

\newpage
\subsection{7-2}
试根据公式 $p = - \sum_{l} a_{l} \frac{\partial \varepsilon_{l}}{\partial V}$ 证明,对于极端相对论粒子
$$\varepsilon = cp = c \frac{2 \pi \hbar}{L} (n_{x}^{2} + n_{y}^{2} + n_{z}^{2})^{\frac{1}{2}}$$
$$(n_{x}, n_{y}, n_{z} = 0, \pm 1, \pm 2, \cdots ),$$
有
$$p = \frac{1}{3} \frac{U}{V}$$
上述结论对于玻耳兹曼分布、玻色分布和费米分布都成立。

\newpage
\subsection{7-3}
当选择不同的能量零点时,粒子第 $l$ 个能级的能量可以取为 $\varepsilon_l$ 或 $\varepsilon_l^*$。以 $\Delta$ 表示二者之差,$\Delta = \varepsilon_l^* - \varepsilon_l$。试证明相应的配分函数存在以下关系 $Z_i^* = e^{-\beta\Delta}Z_i$,并讨论由配分函数 $Z_i$ 和 $Z_i^*$ 求得的热力学函数有何差别。

\newpage
\subsection{7-4}
试证明,对于遵从玻耳兹曼分布的定域系统,熵函数可以表示为
$$S = -Nk \sum_i P_i \ln P_i,$$
式中 $P_i$ 是粒子处在量子态 $s$ 的概率,
$$P_s = \frac{e^{-\alpha - \beta e_s}}{N} = \frac{e^{-\beta e_s}}{Z_1},$$
$\sum$ 是对粒子的所有量子态求和。

对于满足经典极限条件的非定域系统,熵的表达式有何不同?

\newpage
\subsection{7-5}
固体含有 A, B 两种原子。试证明由于原子在格点上的随机分布引起的混合熵为
$$S = k \ln \frac{N!}{(Nx)! [N(1-x)]!} = -Nk [x \ln x + (1-x) \ln (1-x)] ,$$
其中 $N$ 是总原子数,$x$ 是 A 原子的百分比,$1-x$ 是 B 原子的百分比。注意 $x < 1$,上式给出的熵为正值。

\newpage
\subsection{7-6}
晶体含有 $N$ 个原子。原子在晶体中的正常位置如图 7-1 中的“O”所示。当原子离开正常位置而占据图中的“×”位置时,晶体中就出现空位和间隙原子。晶体的这种缺陷称为弗仑克尔(Frenkel)缺陷。

(a) 假设正常位置和间隙位置都是 $N$,试证明,由于在晶体中形成 n 个空位和间隙原子而具有的熵等于
$$S = 2k \ln \frac{N!}{n! (N-n)!}$$

(b) 设原子在间隙位置和正常位置的能量差为 $u$。试由自由能 $F = nu-TS$ 为极小证明,温度为 $T$ 时,空位和间隙原子数为
$$n \approx Ne^{-\frac{u}{2kT}} \quad (\text{设 } n \ll N)$$

\newpage
\subsection{7-7}
如果原子脱离晶体内部的正常位置而占据表面上的正常位置,构成新的一层,晶体将出现如图7-2所示的缺陷,称为肖特基缺陷。以 $N$ 表示晶体中的原子数,$n$ 表示晶体中的缺陷数。如果忽略晶体体积的变化,试用自由能为极小的条件证明,温度为 $T$ 时,有
$$n \approx Ne^{-\frac{W}{kT}} \quad (\text{设} \ n \ll N),$$
其中 $W$ 为原子在表面位置与正常位置的能量差。

\newpage
\subsection{7-8}
稀薄气体由某种原子组成。原子两个能级能量之差为
$$\varepsilon_2 - \varepsilon_1 = \hbar \omega_0.$$
当原子从高能级 $\varepsilon_2$ 跃迁到低能级 $\varepsilon_1$ 时将伴随着光的发射。由于气体中原子的速度分布和多普勒(Doppler)效应,光谱仪观察到的不是单一频率 $\omega_0$ 的谱线,而是频率的一个分布,称为谱线的多普勒增宽。试求温度为 $T$ 时谱线多普勒增宽的表达式。

\newpage
\subsection{7-9}
气体以恒定速度沿 z 方向作整体运动。试证明:在平衡状态下分子动量的最概然分布为
$$e^{-\alpha - \frac{\beta}{2m} [p_{x}^{2}+p_{y}^{2}+(p_{z}-p_{0})^{2}]} \frac{Vdp_{x} dp_{y} dp_{z}}{h^{3}}$$

\newpage
\subsection{7-10}
气体以恒定速度 $v_0$ 沿 z 方向作整体运动,求分子的平均动能量。

\newpage
\subsection{7-11}
表面活性物质的分子在液面上作二维自由运动,可以看作二维气体。试写出二维气体中分子的速度分布和速率分布,并求平均速率 $\bar{v}$,最概然速率 $v_m$ 和方均根速率 $v_s$。

\newpage
\subsection{7-12}
根据麦克斯韦速度分布律导出两分子的相对速度 $v_r = v_2 - v_1$ 和相对速率 $v_r = |v_r|$ 的概率分布,并求相对速率的平均值 $\bar{v_r}$。

\newpage
\subsection{7-13}
试证明,单位时间内碰到单位面积器壁上,速率介于 $v$ 与 $v + dv$ 之间的分子数为
$$d\Gamma (v) = \pi n \left( \frac{m}{2 \pi k T} \right)^{\frac{3}{2}} e^{-\frac{m}{2kT} v^{2}} v^{3} dv.$$

\newpage
\subsection{7-14}
分子从器壁的小孔射出,求在射出的分子束中,分子的平均速率、方均根速率和平均能量。

\newpage
\subsection{7-15}
体积为 $V$ 的容器保持恒定的温度 $T$,容器内的气体通过面积为 $A$ 的小孔缓慢地漏入周围的真空中,求容器中气体压强降到初始压强的 $\frac{1}{e}$ 所需的时间。

\newpage
\subsection{7-16}
承前 5-2 题。
(a) 证明在温度均匀的情况下,由压强差引起的能量流与物质流之比
$$\frac{J_u}{J_n} = \frac{L_{uu}}{L_{nn}} = 2RT.$$

(b) 证明在没有净物质流通过小孔,即 $J_n = 0$ 时,两边的压强差 $\Delta p$ 与温度差 $\Delta T$ 满足:
$$\frac{\Delta p}{\Delta T} = \frac{1}{2} \frac{p}{T},$$
或
$$\frac{p_1}{\sqrt{T_1}} = \frac{p_2}{\sqrt{T_2}}.$$

\newpage
\subsection{7-17}
已知粒子遵从经典玻耳兹曼分布,其能量表达式为
$$\varepsilon = \frac{1}{2m}(p_x^2 + p_y^2 + p_z^2) + ax^2 + bx,$$
其中 $a, b$ 是常量,求粒子的平均能量。

\newpage
\subsection{7-18}
以 $\varepsilon(q_1, \cdots, q_r; p_1, \cdots, p_r)$ 表示玻耳兹曼系统中粒子的能量,试证明
$$\overline{x_i \frac{\partial \varepsilon}{\partial x_j}} = \delta_{ij} kT,$$
其中 $x_i, x_j$ 分别是 $2r$ 个广义坐标和动量中的任意一个,上式称为广义能量均分定理。

\newpage
\subsection{7-19}
非谐振子的能量为
$$\varepsilon = \frac{1}{2m} p_x^2 + \frac{m\omega^2}{2}x^4,$$
试根据广义能量均分定理求振子的平均能量。

\newpage
\subsection{7-20}
已知极端相对论粒子的能量动量关系为
$$\varepsilon = c(p_x^2 + p_y^2 + p_z^2)^{\frac{1}{2}}.$$
假设由近独立、极端相对论粒子组成的气体满足经典极限条件,试由广义能量均分定理求粒子的平均能量。

\newpage
\subsection{7-21}
气柱的高度为 $H$,处在重力场中。试证明此气柱的内能和热容为
$$U = U_0 + NkT - \frac{NmgH}{e^{\frac{mgH}{kT}} - 1},$$
$$C_v = C_v^0 + Nk - \frac{N(mgH)^2 e^{\frac{mgH}{kT}}}{(e^{\frac{mgH}{kT}} - 1)^2} \frac{1}{kT^2}.$$

\newpage
\subsection{7-22}
试求双原子分子理想气体的振动熵。

\newpage
\subsection{7-23}
对于双原子分子,常温下$kT$远大于转动的能级间距。试求双原子分子理想气体的转动熵。

\newpage
\subsection{7-24}
如果原子基态的自旋角动量 $S$ 和轨道角动量 $L$ 不等于零,自旋-轨道耦合作用将导致原子能级的精细结构。考虑能级的精细结构后,电子运动的配分函数为
$$Z_1^e = \sum_j (2J+1) e^{-\frac{\varepsilon_j}{kT}},$$
其中 $\varepsilon_j$ 表示精细结构能级,$J$ 是原子的总角动量量子数,$2J+1$ 是能级 $\varepsilon_j$ 的简并度。试讨论电子运动对单原子理想气体热力学函数的影响。

\newpage
\subsection{7-25}
在温度足够高时,需要计及双原子分子振动的非简谐修正,振动能量的经典表达式为
$$\varepsilon^v = \frac{1}{2m_{\mu}} p^2 + \frac{K}{2} q^2 - bq^3 + cq^4 \quad (b,c 为正)$$
式中最后两项是非简谐修正项,其大小远小于前面两项。试证明,双原子分子气体的振动内能和热容可表示为
$$U' = NkT + Nk^2 T^2 \delta,$$
$$C_v' = Nk + 2Nk^2 T \delta,$$
其中
$$\delta = \frac{15}{2} \frac{b^2}{K^3} - \frac{3c}{K^2},$$
并证明两核的平均距离 $\bar{r}$ 与温度有关,
$$\bar{r} = r_0 + \frac{3b}{K^2}kT,$$
$r_0$ 是两核的平衡间距。

\newpage
\subsection{7-26}
试求爱因斯坦固体的熵。

\newpage
\subsection{7-27}
定域系统含有 $N$ 个近独立粒子,每个粒子有两个非简并能级 $\varepsilon_0$ 和 $\varepsilon_1 (\varepsilon_1 > \varepsilon_0)$。求在温度为 $T$ 的热平衡状态下粒子在两能级的分布,以及系统的内能和熵。讨论在低温和高温极限下的结果。

\newpage
\subsection{7-28}
以 $n$ 表示晶体中原子的密度。设原子的总角动量量子数为 1。在外磁场 $\mathscr{B}$ 下原子磁矩可以有三种不同的取向,即平行、垂直、反平行于外磁场,假设磁矩之间的相互作用可以忽略,试求温度为 $T$ 时晶体的磁化强度 $\mathscr{M}$ 及其在弱场高温极限和强场低温极限下的近似值。

\newpage
\subsection{7-29}
顺磁固体 $(Gd)_2 (SO_4)_3 \cdot 8H_2O$ 的顺磁性来自 $Gd^{3+}$ 离子。$Gd^{3+}$ 离子基态的谱项为 $^8S_{\frac{7}{2}} (L = 0, J = S = \frac{7}{2})$。试求在高温和低温极限下 $(Gd)_2 (SO_4)_3 \cdot 8H_2O$ 的磁化率。

\newpage
\subsection{7-30}
双原子理想气体分子具有固有的电偶极矩 $d_0$,在电场 $\mathscr{E}$ 下转动能量的经典表达式为
$$\varepsilon^r = \frac{1}{2I} \left( p_\theta^2 + \frac{1}{\sin^2 \theta} p_\phi^2 \right) - d_0 \mathscr{E} \cos \theta.$$
证明在经典近似下转动配分函数 $Z_1^r$ 为
$$Z_1^r = \frac{I}{\beta h^2} \frac{e^{\beta d_0 \mathscr{E}} - e^{-\beta d_0 \mathscr{E}}}{\beta d_0 \mathscr{E}}.$$