\section{习题}

\newpage
\subsection{7.1}
试根据公式 $p = -\sum_i a_i \frac{\partial \epsilon_i}{\partial V}$ 证明,对于非相对论粒子
$$\epsilon = \frac{p^2}{2m} = \frac{1}{2m} \left( \frac{2\pi h}{L} \right)^2 \left( n_x^2 + n_y^2 + n_z^2 \right), \quad (n_x, n_y, n_z = 0, \pm 1, \pm 2, \cdots)$$
有
$$p = \frac{2U}{3V}$$
上述结论对于玻耳兹曼分布、玻色分布和费米分布都成立。

\newpage
\subsection{7.2}
试根据公式 $ p = -\sum_i a_i \frac{\partial \varepsilon_i}{\partial V} $ 证明,对于相对论粒子

$$\varepsilon = cp = c \frac{2\pi\hbar}{L} \left( n_x^2 + n_y^2 + n_z^2 \right)^{\frac{1}{2}}, \quad (n_x, n_y, n_z = 0, \pm 1, \pm 2, \cdots),$$

有

$$p = \frac{1}{3} \frac{U}{V}.$$

上述结论对于玻耳兹曼分布、玻色分布和费米分布都成立。

\newpage
\subsection{7.3}
当选择不同的能量零点时,粒子第 $ i $ 个能级的能量可以取为 $ \varepsilon_i $ 或 $ \varepsilon_j $。以 $\Delta$ 表示二者之差,$\Delta = \varepsilon_i - \varepsilon_j$。试证明相应配分函数存在以下关系 $ Z_1^* = e^{-\beta\Delta}Z_1 $,并讨论由配分函数 $ Z_1 $ 和 $ Z_1^* $ 求得的热力学函数有何差别。

\newpage
\subsection{7.4}
试证明,对于遵从玻耳兹曼分布的定域系统,熵函数可以表示为

$$S = -Nk \sum_s P_s \ln P_s,$$

式中 $ P_s $ 是粒子处在量子态 $ s $ 的概率,

$$P_s = \frac{e^{-\alpha - \beta \varepsilon_s}}{N} = \frac{e^{-\beta \varepsilon_s}}{Z_1},$$

$\sum_s$ 是对粒子的所有量子态求和。

对于满足经典极限条件的非定域系统,熵的表达式有何不同?

\newpage
\subsection{7.5}
因体含有A、B两种原子。试证明由于原子在晶体格点的随机分布引起的混合熵为
$$S = k \ln \frac{N!}{(Nx)! [N(1-x)]!} = -Nk \left[ x \ln x + (1-x) \ln (1-x) \right],$$
其中 $ N $ 是总原子数,$ x $ 是A原子的百分比,$ 1-x $ 是B原子的百分比。注意 $ x < 1 $,上式给出的熵为正值。

\newpage
\subsection{7.6}
晶体含有 $N$ 个原子。原子在晶体中的正常位置如图中的“0”所示。当原子离开正常位置而占据图中的“×”位置时,晶体中就出现缺位和填隙原子。晶体的这种缺陷称为弗伦克尔(Frenkel)缺陷。

(a) 假设正常位置和填隙位置都是 $N$,试证明,由于在晶体中形成 $n$ 个缺位和填隙原子而具有的熵等于
$$ S = 2k \ln \frac{N!}{n!(N-n)!} $$

(b) 设原子在填隙位置和正常位置的能量差为 $u$。试由自由能 $F = nu-TS$ 为极小证明,温度为 $T$ 时,缺位和填隙原子数为
$$ n \approx Ne^{\frac{u}{2kT}} \quad (\text{设 } n \ll N) $$

\newpage
\subsection{7.7}
如果原子脱离晶体内部的正常位置而占据表面上的正常位置,构成新的一层,晶体将出现如图所示的缺陷,称为肖脱基缺陷。以 $ N $ 表示晶体中的原子数,$ n $ 表示晶体中的缺陷数。如果忽略晶体体积的变化,试用自由能为极小的条件证明,温度为 $ T $ 时,有
$$ n \approx Ne^{\frac{W}{kT}} \quad (\text{设 } n << N) $$
其中 $ W $ 为原子在表面位置与正常位置的能量差。

\newpage
\subsection{7.8}
稀薄气体由某种原子组成。原子两个能级能量之差为 $$ \varepsilon_2 - \varepsilon_1 = h\omega_0. $$

当原子从高能级 $\varepsilon_2$ 跃迁到低能级 $\varepsilon_1$ 时将伴随着光的发射。由于气体中原子的速度分布和多普勒(Doppler)效应,光谱仪观察到的不是单一频率 $\omega_0$ 的谱线,而是频率的一个分布,称为谱线的多普勒增宽。试求温度为 $T$ 时谱线多普勒增宽的表达式。

\newpage
\subsection{7.9}
气体以恒定速度沿z方向作整体运动。试证明:在平衡状态下分子动量的最概然分布为

$$e^{-a z} \frac{\rho}{2m} \left[ p_z^2 + p_y^2 + (p_z - p_0)^2 \right] \frac{Vdp_z dp_y dp_z}{h^3}.$$

\newpage
\subsection{7.10}
气体以恒定速度 $ v_{0} $ 沿 z 方向作整体运动,求分子的平均动能量。

\newpage
\subsection{7.11}
表面活性物质的分子在液面上作二维自由运动,可以看作二维气体。试写出二维气体中分子的速度分布和速率分布,并求平均速率 $\overline{v}$、最概然速率 $v_m$ 和方均根速率 $v_s$。

\newpage
\subsection{7.12}
根据麦克斯韦速度分布律导出两分子的相对速度 $ v_r = v_2 - v_1 $ 和相对速率 $ v_t = |v_t| $ 的概率分布,并求相对速率的平均值 $ \overline{v_r} $。

\newpage
\subsection{7.13}
试证明,单位时间内碰到单位面积器壁上,速率介于 $v + dv$ 之间的分子数为
$$ d\Gamma (v) = \pi n \left( \frac{m}{2\pi kT} \right)^{\frac{3}{2}} e^{-\frac{mv^2}{2kT}} v^3 dv $$

\newpage
\subsection{7.14}
分子从器壁的小孔射出,求在射出的分子束中,分子的平均速率、方均根速率和平均能量。

\newpage
\subsection{7.15}
承前5.2题。
(a) 证明在温度均匀的情况下,由压强差引起的能量流与物质流之比
$$\frac{J_u}{J_n} = \frac{L_m}{L_m} = 2RT.$$

(b) 证明在没有净物质流通过小孔,即 $J_u = 0$ 时,两边的压强差 $\Delta p$ 与温度差 $\Delta T$ 满足:
$$\frac{\Delta p}{\Delta T} = \frac{1}{2} \frac{p}{T}.$$

或
$$\frac{p_1}{\sqrt{T_1}} = \frac{p_2}{\sqrt{T_2}}.$$

\newpage
\subsection{7.16}
已知粒子遵从经典玻耳兹曼分布,其能量表达式为

$$ \varepsilon = \frac{1}{2m} \left( p_x^2 + p_y^2 + p_z^2 \right) + ax^2 + bx, $$

其中 $ a, b $ 是常量,求粒子的平均能量。

\newpage
\subsection{7.17}
气柱的高度为 $ H $,处在重力场中。试证明此气柱的内能和热容量为

$$ U = U_0 + NkT - \frac{NmgH}{\frac{mgH}{e^{kT}} - 1}, $$

$$ C_v = C_v^0 + Nk - \frac{N(mgh)^2 e^{\frac{mgH}{kT}}}{(e^{\frac{mgH}{kT}}-1)^2} \cdot \frac{1}{kT^2}. $$

\newpage
\subsection{7.18}
试求双原子分子理想气体的振动熵。

\newpage
\subsection{7.19}
对于双原子分子,常温下 $kT$ 远大于转动的能级间距。试求双原子分子理想气体的转动熵。

\newpage
\subsection{7.20}
试求爱因斯坦固体的熵。

\newpage
\subsection{7.21}
定域系统含有 $N$ 个近独立粒子,每个粒子有两个非简并能级 $\epsilon_0$ 和 $\epsilon_1 (\epsilon_1 > \epsilon_0)$。求在温度为 $T$ 的热平衡状态下粒子在两能级的分布,以及系统的内能和熵。讨论在低温和高温极限下的结果。

\newpage
\subsection{7.22}
以 $ n $ 表示晶体中原子的密度。设原子的总角动量量子数为1,磁矩为 $\mu$。在外磁场 $ B $ 下原子磁矩可以有三个不同的取向,即平行、垂直、反平行于外磁场。假设磁矩之间的相互作用可以忽略。试求温度为 $ T $ 时晶体的磁化强度 $ M $ 及其在弱磁场高温极限和强场低温极限下的近似值。

\newpage
\subsection{7.23}
气体分子具有固有的电偶极矩 $ d_0 $,在电场 $ E $ 下转动能量的经典表达式为
$$\varepsilon' = \frac{1}{2I} \left( p_\theta^2 + \frac{1}{\sin^2 \theta} p_\varphi^2 \right) - d_0 E \cos \theta.$$
证明在经典近似下转动配分函数
$$Z_i = \frac{1}{\beta h^2} \frac{e^{\beta d_0 E} - e^{-\beta d_0 E}}{\beta d_0 E}.$$

\newpage
\subsection{7.24}
承上题。试证明在高温极限 $(\beta d_0 E << 1)$ 下,单位体积的电偶极矩(电极化强度)为

$$ P = \frac{n d_0^2}{3kT} E. $$

\newpage
\subsection{7.25}
$^3$He是费米子,其自旋为1/2在液$^3$He中原子有很强的相互作用。根据朗道的正常费米液体理论,可以将液$^3$He看作是由与原子数目相同的$^3$He准粒子构成的费米液体。已知液$^3$He的密度为 $81 \, \text{kg} \cdot \text{m}^{-3}$,在0.1K以下的定容热容量为 $C_v = 2.89 \, \text{NkT}$。试估算$^3$He准粒子的有效质量 $m^*$。

\newpage
\subsection{补充题1}
试根据麦克斯韦速度分布律证明,速率和平均能量的涨落为

$$ \left( \overline{v - v} \right)^2 = \frac{kT}{m} \left( 3 - \frac{8}{\pi} \right), $$
$$ \left( \varepsilon - \overline{\varepsilon} \right)^2 = \frac{3}{2} (kT)^2. $$

\newpage
\subsection{补充题2}
体积为 $V$ 的容器保持恒定的温度 $T$,容器内的气体通过面积为 $A$ 的小孔缓慢地漏入周围的真空中,求容器中气体压强降到初始压强的 $\frac{1}{e}$ 所需的时间。

\newpage
\subsection{补充题3}
以 $ \varepsilon(q_1, \cdots, q_r; p_1, \cdots, p_r) $ 表示玻耳兹曼系统中粒子的能量,试证明
$$ x_i \frac{\partial \varepsilon}{\partial x_j} = \delta_{ij} kT, $$

其中 $ x_i, x_j $ 分别是 $ 2r $ 个广议坐标和动量中的任意一个,上式称为广义能量均分定理。

\newpage
\subsection{补充题4}
已知极端相对论粒子的能量-动量关系为
$$ \varepsilon = c \left( p_x^2 + p_y^2 + p_z^2 \right)^{\frac{1}{2}}. $$
假设由近独立、极端相对论粒子组成的气体满足经典极限条件,试由广义能量均分定理求粒子的平均能量。

\newpage
\subsection{补充题5}
如果原子基态的自旋角动量S和轨道角动量L不等于零,自旋-轨道耦合作用将导致原子能级的精细结构。考虑能级的精细结构后,电子运动的配分函数为

$$ Z_i^e = \sum_j (2J+1) e^{-\frac{\varepsilon_j}{kT}}, $$

其中$\varepsilon_j$表示精细结构能级,$J$是原子的总角动量量子数,$2J+1$是能级$\varepsilon_j$的简并度。试讨论电子运动对单原子理想气体热力学函数的影响。

\newpage
\subsection{补充题6}
在温度足够高时,需要计及双原子分子振动的非简谐修正,振动能量的经典形式为

$$ \varepsilon^v = \frac{1}{2\mu} p^2 + \frac{K}{2} q^2 - bq^3 + cq^4 \quad (b, c 为正), $$

式中最后两项是非简谐修正项,其大小远小于前面两项。试证明,双原子分子气体的振动内能和热容量可表示为

$$ U^v = NkT + Nk^2 T^2 \delta, $$
$$ C_v^v = Nk + 2Nk^2 T \delta, $$

其中

$$ \delta = \frac{15}{2} \frac{b^2}{K^3} - \frac{3c}{K^2}, $$

并证明两核的平均距离 $\bar{r}$ 与温度有关,

$$ \bar{r} = r_0 + \frac{3b}{K^2} kT, $$

$r_0$ 是两核的平衡间距。

\newpage
\subsection{补充题7}
顺磁固体 $(\text{Gd})_2 (\text{SO}_4)_3 \cdot 8\text{H}_2\text{O}$ 的顺磁性来自 $\text{Gd}^{3+}$ 离子。$\text{Gd}^{3+}$ 离子基态的谱项为

$$ s_{\frac{1}{2}} L = 0, \, J = S = \frac{7}{2}. $$

试求在高温和低温极限下 $(\text{Gd})_2 (\text{SO}_4)_3 \cdot 8\text{H}_2\text{O}$ 的磁化率。