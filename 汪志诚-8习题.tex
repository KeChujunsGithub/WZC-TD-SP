\section{8}

\newpage
\subsection{8.1}
试证明,对于玻色或费米统计,玻耳兹曼关系成立,即
$$ S = k \ln \Omega. $$

\newpage
\subsection{8.2}
试证明,理想玻色和费米系统的熵可分别表示为
$$ S_{BE} = k \sum_s \left[ f_s \ln f_s - (1 + f_s) \ln (1 + f_s) \right], $$
$$ S_{FD} = -k \sum_s \left[ f_s \ln f_s + (1 - f_s) \ln (1 - f_s) \right], $$
其中 $f_s$ 为量子态上的平均粒子数。$\sum_s$ 表示对粒子的所有量子态求和。同时证明,当 $f_s \ll 1$ 时,有
$$ S_{BE} \approx S_{FD} \approx S_{MB} = -k \sum_s (f_s \ln f_s - f_s). $$

\newpage
\subsection{8.3}
求弱简并理想费米(玻色)气体的压强和熵。

\newpage
\subsection{8.4}
试证明,在热力学极限下均匀的二维理想玻色气体不会发生玻色-受因斯坦凝聚。

\newpage
\subsection{8.5}
约束在磁光陷阱中的原子,在三维谐振势场
$$ V = \frac{1}{2} m (\omega_x^2 x^2 + \omega_y^2 y^2 + \omega_z^2 z^2) $$
中运动。如果原子是玻色子,试证明:在 $T \leq T_c$ 时将有宏观量级的原子凝聚在能量为
$$ \varepsilon_0 = \frac{\hbar}{2} (\omega_x + \omega_y + \omega_z) $$
的基态,在 $N \to \infty, \overline{\omega} \to 0, N \overline{\omega}^3$ 保持有限的热力学极限下,临界温度 $T_c$ 由下式确定:
$$ N = 1.202 \times \left( \frac{kT_c}{\hbar \omega} \right)^3, $$
其中 $\overline{\omega} = (\omega_x \omega_y \omega_z )^{\frac{1}{3}}$。温度为 $T$ 时凝聚在基态的原子数 $N_0$ 与总原子数 $N$ 之比为
$$ \frac{N_0}{N} = 1 - \left( \frac{T}{T_c} \right)^3. $$

\newpage
\subsection{8.6}
承前 8.5 题,如果 $\omega_z \gg \omega_x, \omega_y$,则在 $kT \ll \hbar \omega_z$ 的情形下,原子在 z 方向的运动将冻结在基态作零点振动,于是形成二维原子气体。试证明 $T < T_c$ 时原子的二维运动中将有宏观量级的原子凝聚在能量为 $\varepsilon_0 = \frac{\hbar}{2} (\omega_x + \omega_y)$ 的基态,在 $N \to \infty$,$\omega \to 0$,$N \omega^2$ 保持有限的热力学极限下,临界温度 $T_c$ 由下式确定:
$$ N = 1.645 \left( \frac{kT_c}{\hbar \omega} \right)^2, $$
其中 $\omega = (\omega_x \omega_y )^{\frac{1}{2}}$。温度为 $T$ 时凝聚在基态的原子数 $N_0$ 与总原子数 $N$ 之比为
$$ \frac{N_0}{N} = 1 - \left( \frac{T}{T_c} \right)^2. $$

\newpage
\subsection{8.7}
计算温度为 $T$ 时,在体积 $V$ 内光子气体的平均总光子数,并据此估算\\
(a) 温度为 1000K 的平衡辐射。\\
(b) 温度为 3K 的宇宙背景辐射中光子的数密度。

\newpage
\subsection{8.8}
试根据普朗克公式证明平衡辐射内能密度按波长的分布为
$$ u(\lambda, T)d\lambda = \frac{8\pi hc}{\lambda^5} \frac{d\lambda}{e^{\frac{\hbar c}{\lambda kT}}-1}, $$
并据此证明,使辐射内能密度取极大的波长 $\lambda_m$ 满足方程
$$ \left( x = \frac{hc}{\lambda_m kT} \right) $$
$$ 5e^{-x} + x = 5. $$
这个方程的数值解为 $x = 4.9651$。因此
$$ \lambda_m T = \frac{hc}{4.9651k}, $$
$\lambda_m$ 随温度增加向短波方向移动。

\newpage
\subsection{8.9}
按波长分布太阳辐射能的极大值在 $\lambda \approx 480 \mathrm{nm}$ 处,假设太阳是黑体,求太阳表面的温度。

\newpage
\subsection{8.10}
试根据热力学公式 $S = \int \frac{C_v}{T} dT$ 及光子气体的热容量求光子气体的熵。

\newpage
\subsection{8.11}
试计算平衡辐射中单位时间碰到单位面积器壁上的光子所携带的能量,由此即得平衡辐射的通量密度 $J_u$。计算6000K和1000K时 $J_u$ 的值。

\newpage
\subsection{8.12}
室温下某金属中自由电子气体的数密度 $n = 6 \times 10^{28} \mathrm{m}^{-3}$,某半导体中导电电子的数密度为 $n = 10^{28} \mathrm{m}^{-3}$,试验证这两种电子气体是否为简并气体。

\newpage
\subsection{8.13}
银的导电电子数密度为 $5.9 \times 10^{28} \mathrm{m}^{-3}$ 试求 0 K 时电子气体的费米能量、费米速率和简并压。

\newpage
\subsection{8.14}
试求绝对零度下自由电子气体中电子的平均速率。

\newpage
\subsection{8.15}
试证明,在绝对零度下自由电子的碰壁数可表示为
$$ \Gamma = \frac{1}{4} nv, $$
其中 $n = \frac{N}{V}$ 是电子的数密度,$\bar{v}$ 是平均速率。

\newpage
\subsection{8.16}
已知声速 $a = \sqrt{\left( \frac{\partial p}{\partial \rho} \right)_s}$ (式(1.8.8)),试证明在 0 K 理想费米气体中
$$ a = \frac{v_F}{\sqrt{3}}. $$

\newpage
\subsection{8.17}
等温压缩系数 $\kappa_T$ 和绝热压缩系数 $\kappa_S$ 的定义分别为
$$ \kappa_T = -\frac{1}{V}\left(\frac{\partial V}{\partial p}\right)_T $$
和
$$ \kappa_S = -\frac{1}{V}\left(\frac{\partial V}{\partial p}\right)_S. $$
试证明,对于 $0 K$ 的理想费米气体,有
$$ \kappa_T(0) = \kappa_S(0) = \frac{3}{2}\frac{1}{n\mu(0)}. $$

\newpage
\subsection{8.18}
试求在极端相对论条件下自由电子气体在 0K 时的费米能量、内能和简并压。

\newpage
\subsection{8.19}
假设自由电子在二维平面上运动,面密度为 $n$。试求0 K时二维电子气体的费米能量、内能和简并压。

\newpage
\subsection{8.20}
已知0 K时铜中自由电子气体的化学势
$$ \mu(0) = 7.04\mathrm{eV}, $$
试求300 K时的一级修正值。

\newpage
\subsection{8.21}
试根据热力学公式 $S = \int \frac{C_v}{T} dT$,求低温下金属中自由电子气体的熵。

\newpage
\subsection{8.22}
由 $N$ 个自旋极化的粒子组成的理想费米气体处在径向频率为 $\omega_r$,轴向频率为 $\lambda \omega_r$ 的磁光陷阱内,粒子的能量(哈密顿量)为
$$ \varepsilon = \frac{1}{2m} \left( p_x^2 + p_y^2 + p_z^2 \right) + \frac{m}{2} \omega_r^2 \left( x^2 + y^2 + \lambda^2 z^2 \right) $$
试求 $0 K$ 时费米气体的化学势(以费米温度表示)和粒子的平均能量。假设 $N = 10^5$,$\omega_r = 3800\mathrm{s}^{-1}$,$\lambda^2 = 8$,求出数值结果。

\newpage
\subsection{8.23}
承上题,试求低温极限 $T \ll T_F$ 和高温极限 $T \gg T_F$ 下,磁光陷阱中理想费米气体的化学势、内能和热容量。

\newpage
\subsection{8.24}
关于原子核半径 $R$ 的经验公式给出
$$ R = (1.3 \times 10^{-15} \mathrm{m}) \cdot A^{1/3}, $$
式中 $A$ 是原子核所含核子数。假设质子数和中子数相等,均为 $A/2$,试计算二者在核内的密度 $n$。如果将核内的质子和中子看作简并费米气体,试求二者的 $\mu(0)$ 以及核子在核内的平均能量。核子质量 $m_n = 1.67 \times 10^{-27} \mathrm{kg}$。

\newpage
\subsection{8.25}
$^3$He是费米子,其自旋为1/2在液$^3$He中原子有很强的相互作用。根据朗道的正常费米液体理论,可以将液$^3$He看作是由与原子数目相同的$^3$He准粒子构成的费米液体。已知液$^3$He的密度为 $81 \mathrm{kg} \cdot \mathrm{m}^{-3}$,在0.1K以下的定容热容量为 $C_v = 2.89 NkT$。试估算$^3$He准粒子的有效质量 $m^*$。

\newpage
\subsection{补充题1}
写出二维空间中平衡辐射的普朗克公式,并据此求平均总光子数、内能和辐射通量密度。

\newpage
\subsection{补充题2}
金属中的自由电子在外磁场下显示微弱的顺磁性。这是泡利(Pauli)根据费米分布首先从理论上预言的,称为泡利顺磁性。试根据费米分布导出OK金属中自由电子的磁化率。

\newpage
\subsection{补充题3}
金属中的自由电子可以近似看作处在一个恒定势阱中的自由粒子。下图示意地表示 OK 时处在势阱中的电子。$x$ 表示势阱的深度,它等于将处在最低能级 $\varepsilon = 0$ 的电子移到金属外所需的最小功。$\mu(0)$ 表示 OK 时电子气体的化学势。如果将处在费米能级 $\varepsilon = \mu(0)$ 的电子移到金属外,所需的最小功为
$$ W = x - \mu(0) $$
$W$ 称为功函数。$W$ 的大小视不同金属而异,一般是电子伏的量级。高温下处在费米分布中高能量的电子有可能从金属表面逸出。试证明,单位时间内通过金属的单位面积发射的热电流密度为
$$ J = AT^2 e^{-\frac{W}{kT}} $$
上式称为理查孙(Richardson)公式。

\newpage
\subsection{补充题4}
在高纯度的半导体中电子的能量本征值形成图所示的能带结构。OK时价带中的状态完全被电子占据。而导带中的状态则完全未被占据。价带与导带之间有能量为 $\varepsilon_g$ 的能隙,称为禁带,其中不存在电子的可能状态。OK下具有这种能带结构的晶体形成绝缘体。在较高温度下,价带中有些电子因热激发会跃迁到导带,而在价带留下空穴。跃迁到导带的电子和价带中的空穴都参与导电,晶体就形成半导体。这样的半导体称为本征半导体。\\
试证明,温度 $T$ 时本征半导体中电子和空穴的浓度 $n_e$ 和 $n_h$ 为
$$ n_e = n_h = 2\left(\frac{2\pi mkT}{h^2}\right)^{\frac{3}{2}} e^{-\frac{\varepsilon_g}{2kT}} $$

\newpage
\subsection{补充题5}
试求低温下金属中自由电子气体巨配分函数的对数,从而求电子气体的内能、压强和熵。