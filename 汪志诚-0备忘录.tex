\section{备忘录}

\subsection{三大分布}
麦克斯韦-玻尔兹曼分布
$$a_l=\frac{\omega _l}{e^{\alpha +\beta \varepsilon _l}}=\omega _le^{-\alpha -\beta \varepsilon _l}
$$
玻色-爱因斯坦分布
$$a_l=\frac{\omega _l}{e^{\alpha +\beta \varepsilon _l}-1}$$
费米-狄拉克分布
$$a_l=\frac{\omega _l}{e^{\alpha +\beta \varepsilon _l}+1}$$



$$
a_l=\frac{\omega _l}{e^{\alpha +\beta \varepsilon _l}+b}\left\{ \begin{aligned}
	&b=1, \,\,\,\,\mathrm{Fermi-Dirac}\, \mathrm{Distribution}\\
	&b=0, \,\,\,\,\mathrm{Maxwell-Boltzmann} \,\mathrm{Distribution}\\
	&b=-1, \mathrm{Bose-Einstein} \,\mathrm{Distribution}\\
\end{aligned} \right. 
$$


%%%%%%%%%%%%%%%%%%%%%%%%%%%%%%%%%%%%%%%%%%%%%%%%%%%%%%%%%%%%%%%%%%%%%%%%
\subsection{总微观状态数}
玻尔兹曼系统总微观状态数$$\Omega _{\mathrm{M}.\mathrm{B}.}=\frac{N!}{\prod_l{a_l!}}\prod_l{\omega _{l}^{a_l}}
$$
玻色系统总微观状态数$$\Omega _{\mathrm{B}.\mathrm{E}.}=\prod_l{\frac{\left( \omega _l+a_l-1 \right) !}{a_l!\left( \omega _l-1 \right) !}}
$$
费米系统总微观状态数$$\Omega _{\mathrm{F}.\mathrm{D}.}=\prod_l{\frac{\omega _l!}{a_l!\left( \omega _l-a_l \right) !}}
$$


%%%%%%%%%%%%%%%%%%%%%%%%%%%%%%%%%%%%%%%%%%%%%%%%%%%%%%%%%%%%%%%%%%%%%%%%5
\subsection{系统}
\begin{tabularx}{\textwidth}{|l|X|}
\hline
\textbf{系统} & \textbf{描述} \\ 
\hline
玻尔兹曼系统 & 
\makecell[l]{粒子可以分辨,每一个量子态能容纳的粒子数不受限制\\(由可分辨的全同近独立粒子组成的)} \\
\hline
\makecell[l]{玻色-爱因斯坦系统\\(玻色系统)} & 
\makecell[l]{粒子不可分辨,每一个量子态能容纳的粒子数不受限制\\(每一个量子态可以容纳多个粒子)\\(处在以一个个体量子态上的粒子数不受限制)} \\
\hline
\makecell[l]{费米-狄拉克系统\\(费米系统)} & 
\makecell[l]{粒子不可分辨,每一个量子态能容纳的粒子数只能为一\\(每一个量子态只能容纳一个粒子)\\(处在以一个个体量子态上的粒子数只能为一)} \\
\hline
\end{tabularx}

表格二
\\  %增加两个斜线,防止首行缩进
\begin{tabularx}{\textwidth}{|l|X|}
\hline
\textbf{系统} & \textbf{描述} \\ 
\hline
经典系统 & 
全同粒子可以分辨,粒子轨道确定,粒子交换前后系统的运动状态不同(改变) \\
\hline
量子系统 & 
全同粒子不可分辨,波粒二象性,粒子交换前后系统的运动状态相同(不变) \\
\hline
\end{tabularx}





