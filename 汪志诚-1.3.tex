\section{物态方程}



\subsection{笔记:物态方程}
物态方程:给出温度与状态参量之间的函数关系的方程


简单系统的物态方程
\begin{equation}
    f(p,V,T)=0
\end{equation}

式(1.3.1)的具体函数关系视不同的物质而异.
由于 $p$、$V$、$T$之间存在这一函数关系,在实际问题中,我们可以根据方便将其中两个量看作独立参量,而将第三个量看作这两个量的函数.
例如,
若将 $V$ 和 $T$看作独立参量,$p$ 便是它们的函数;
若将 $T$ 和 $p$看作独立参量,$V$ 便是它们的函数.
若将 $p$ 和 $V$看作独立参量,$T$ 便是它们的函数.

\subsection{笔记:与物态方程相关的物理量}
体膨胀系数:描述在压强保持不变的条件下,温度升高 1 K 所引起的物体体积的相对变化
\begin{equation}
    \alpha = \frac{1}{V}\left( \frac{\partial V}{\partial T}\right)_{p} 
\end{equation}
压强系数:描述在体积保持不变的条件下,温度升高1K所引起的物体压强的相对变化
\begin{equation}
     \beta = \frac{1}{p}\left( \frac{\partial p}{\partial T}\right)_{V} 
\end{equation}
等温压缩系数:在温度保持不变的条件下,增加单位压强所引起的物体体积的相对变化
\begin{equation}
     \kappa_{T} = - \frac{1}{V}\left( \frac{\partial V}{\partial p}\right)_{T}
\end{equation}
由于,偏导数关系
\begin{equation}
     \left(\frac{\partial V}{\partial p}\right)_T\left(\frac{\partial p}{\partial T}\right)_V\left(\frac{\partial T}{\partial V}\right)_p=-1
\end{equation}
三个物理量之间的关系
\begin{equation}
    \alpha=\kappa_T\beta p
\end{equation}

\newpage
\subsection{推导:}
参考附录
考虑状态函数 
\begin{equation}
    f(p, V, T) = 0
\end{equation}
计算全微分
\begin{equation}
\mathrm{d}f=\frac{\partial f}{\partial p}\mathrm{d}p+\frac{\partial f}{\partial V}\mathrm{d}V+\frac{\partial f}{\partial T}\mathrm{d}T
\end{equation}
且全微分为0
\begin{equation}
\mathrm{d}f=0
\end{equation}
得到
\begin{equation}
\frac{\partial f}{\partial p}\mathrm{d}p+\frac{\partial f}{\partial V}\mathrm{d}V+\frac{\partial f}{\partial T}\mathrm{d}T=0
\end{equation}
当$\mathrm{d}T=0$
得到
\begin{equation}
\begin{aligned}
\frac{\partial f}{\partial p}\mathrm{d}p+\frac{\partial f}{\partial V}\mathrm{d}V=0
\\
\frac{\partial f}{\partial V}\mathrm{d}V=-\frac{\partial f}{\partial p}\mathrm{d}p
\end{aligned}
\end{equation}
得到
\begin{equation}
\left( \frac{\partial V}{\partial p} \right) _T=-\frac{\left( \frac{\partial f}{\partial p} \right) _{V,T}}{\left( \frac{\partial f}{\partial V} \right) _{T,p}}
\end{equation}
当$\mathrm{d}V=0$
得到
\begin{equation}
\begin{aligned}
\frac{\partial f}{\partial T}\mathrm{d}T+\frac{\partial f}{\partial p}\mathrm{d}p=0
\\
\frac{\partial f}{\partial p}\mathrm{d}p=-\frac{\partial f}{\partial T}\mathrm{d}T
\end{aligned}
\end{equation}
得到
\begin{equation}
\left( \frac{\partial p}{\partial T} \right) _V=-\frac{\left( \frac{\partial f}{\partial T} \right) _{p,V}}{\left( \frac{\partial f}{\partial p} \right) _{V,T}}
\end{equation}
当$\mathrm{d}p=0$
得到
\begin{equation}
\begin{aligned}
\frac{\partial f}{\partial T}\mathrm{d}T+\frac{\partial f}{\partial V}\mathrm{d}V=0
\\
\frac{\partial f}{\partial T}\mathrm{d}T=-\frac{\partial f}{\partial V}\mathrm{d}V
\end{aligned}
\end{equation}
得到
\begin{equation}
\left( \frac{\partial T}{\partial V} \right) _p=-\frac{\left( \frac{\partial f}{\partial V} \right) _{T,p}}{\left( \frac{\partial f}{\partial T} \right) _{p,V}}
\end{equation}
综合
\begin{equation}
\begin{aligned}
\left( \frac{\partial V}{\partial p} \right) _T\left( \frac{\partial p}{\partial T} \right) _V\left( \frac{\partial T}{\partial V} \right) _p&=-\frac{{\color[RGB]{240, 0, 0} \left( \frac{\partial f}{\partial p} \right) _{V,T}}}{{\color[RGB]{0, 160, 0} \left( \frac{\partial f}{\partial V} \right) _{T,p}}}\frac{{\color[RGB]{0, 0, 240} \left( \frac{\partial f}{\partial T} \right) _{p,V}}}{{\color[RGB]{240, 0, 0} \left( \frac{\partial f}{\partial p} \right) _{V,T}}}\frac{{\color[RGB]{0, 160, 0} \left( \frac{\partial f}{\partial V} \right) _{T,p}}}{{\color[RGB]{0, 0, 240} \left( \frac{\partial f}{\partial T} \right) _{p,V}}}
\
&=-1
\end{aligned}
\end{equation}

\newpage
\subsection{推导:}
1.根据三个物理量
\begin{equation}
    \begin{aligned}
        \alpha &=\frac{1}{V}\left( \frac{\partial V}{\partial T} \right) _p
\\
\beta &=\frac{1}{p}\left( \frac{\partial p}{\partial T} \right) _V
\\
\kappa _T&=-\frac{1}{V}\left( \frac{\partial V}{\partial p} \right) _T
    \end{aligned}
\end{equation}
写出
\begin{equation}
    \begin{aligned}
        \left( \frac{\partial V}{\partial T} \right) _p&=V\alpha 
\\
\left( \frac{\partial p}{\partial T} \right) _V&=p\beta 
\\
\left( \frac{\partial V}{\partial p} \right) _T&=-V\kappa _T
    \end{aligned}
\end{equation}
2.根据
\begin{equation}
    \left( \frac{\partial V}{\partial p} \right) _T\left( \frac{\partial p}{\partial T} \right) _V\left( \frac{\partial T}{\partial V} \right) _p=-1
\end{equation}
得到
\begin{equation}
    \left( \frac{\partial V}{\partial p} \right) _T\left( \frac{\partial p}{\partial T} \right) _V=-\left( \frac{\partial T}{\partial V} \right) _{p}^{-1}
\end{equation}
即
\begin{equation}
    \left( \frac{\partial V}{\partial p} \right) _T\left( \frac{\partial p}{\partial T} \right) _V=-\left( \frac{\partial V}{\partial T} \right) _p
\end{equation}
3.代入得到
\begin{equation}
    -V\kappa _T\cdot p\beta =-V\alpha 
\end{equation}
即
\begin{equation}
    \alpha =\kappa _T\beta p
\end{equation}



