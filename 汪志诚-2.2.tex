\section{麦克斯韦关系的简单应用}

目标:利用麦克斯韦关系,将无法直接通过实验测量的物理量转化为可直接实验测量的物理量

\newpage
\subsection{推导}
1.内能$U=U(T,V)$的全微分
\begin{equation}
    \mathrm{d}U=\left( \frac{\partial U}{\partial T} \right) _V\mathrm{d}T+\left( \frac{\partial U}{\partial V} \right) _T\mathrm{d}V
\end{equation}
2.热力学基本方程
\begin{equation}
    \mathrm{d}U=T\mathrm{d}S-p\mathrm{d}V
\end{equation}
3.熵$S=S(T,V)$的全微分
\begin{equation}
    \mathrm{d}S=\left( \frac{\partial S}{\partial T} \right) _V\mathrm{d}T+\left( \frac{\partial S}{\partial V} \right) _T\mathrm{d}V
\end{equation}
4.将熵的全微分代入热力学基本方程得到
\begin{equation}
    \begin{aligned}
        \mathrm{d}U&=T\left[ \left( \frac{\partial S}{\partial T} \right) _V\mathrm{d}T+\left( \frac{\partial S}{\partial V} \right) _T\mathrm{d}V \right] -p\mathrm{d}V
\\
&=T\left( \frac{\partial S}{\partial T} \right) _V\mathrm{d}T+\left[ T\left( \frac{\partial S}{\partial V} \right) _T-p \right] \mathrm{d}V
    \end{aligned}
\end{equation}
5.对比得到
\begin{equation}
    \left( \frac{\partial U}{\partial T} \right) _V=T\left( \frac{\partial S}{\partial T} \right) _V
\end{equation}
和
\begin{equation}
    \left( \frac{\partial U}{\partial V} \right) _T=T\left( \frac{\partial S}{\partial V} \right) _T-p
\end{equation}
6.利用定容热容定义
\begin{equation}
    C_V=\left( \frac{\partial U}{\partial T} \right) _V
\end{equation}
和麦克斯韦关系
\begin{equation}
    \left( \frac{\partial S}{\partial V} \right) _T=\left( \frac{\partial p}{\partial T} \right) _V
\end{equation}
得到
\begin{equation}
    C_V=T\left( \frac{\partial S}{\partial T} \right) _V
\end{equation}
且
\begin{equation}
    \left( \frac{\partial U}{\partial V} \right) _T=T\left( \frac{\partial p}{\partial T} \right) _V-p
\end{equation}



\newpage
\subsection{推导:}
1.焓$H=H(T,p)$的全微分
\begin{equation}
    \mathrm{d}H=\left( \frac{\partial H}{\partial T} \right) _p\mathrm{d}T+\left( \frac{\partial H}{\partial p} \right) _T\mathrm{d}p
\end{equation}
2.热力学基本方程
\begin{equation}
    \mathrm{d}H=T\mathrm{d}S+V\mathrm{d}p
\end{equation}
3.熵$S=S(T,p)$的全微分
\begin{equation}
    \mathrm{d}S=\left( \frac{\partial S}{\partial T} \right) _p\mathrm{d}T+\left( \frac{\partial S}{\partial p} \right) _T\mathrm{d}p
\end{equation}
4.将熵的全微分代入热力学基本方程得到
\begin{equation}
    \begin{aligned}
        \mathrm{d}H&=T\left[ \left( \frac{\partial S}{\partial T} \right) _p\mathrm{d}T+\left( \frac{\partial S}{\partial p} \right) _T\mathrm{d}p \right] +V\mathrm{d}p
\\
&=T\left( \frac{\partial S}{\partial T} \right) _p\mathrm{d}T+\left[ T\left( \frac{\partial S}{\partial p} \right) _T+V \right] \mathrm{d}p
    \end{aligned}
\end{equation}
5.对比得到
\begin{equation}
    \left( \frac{\partial H}{\partial T} \right) _p=T\left( \frac{\partial S}{\partial T} \right) _p
\end{equation}
和
\begin{equation}
    \left( \frac{\partial H}{\partial p} \right) _T=T\left( \frac{\partial S}{\partial p} \right) _T+V
\end{equation}
6.利用定容热容定义
\begin{equation}
    C_p=\left( \frac{\partial H}{\partial T} \right) _p
\end{equation}
和麦克斯韦关系
\begin{equation}
    \left( \frac{\partial S}{\partial p} \right) _T=-\left( \frac{\partial V}{\partial T} \right) _p
\end{equation}
得到
\begin{equation}
    C_p=T\left( \frac{\partial S}{\partial T} \right) _p
\end{equation}
且
\begin{equation}
    \left( \frac{\partial H}{\partial p} \right) _T=V-T\left( \frac{\partial V}{\partial T} \right) _p
\end{equation}


\subsection{推导}
定容热容
\begin{equation}
    C_V=\left( \frac{\partial U}{\partial T} \right) _V=T\left( \frac{\partial S}{\partial T} \right) _V
\end{equation}
定容热容
\begin{equation}
    C_p=\left( \frac{\partial H}{\partial T} \right) _p=T\left( \frac{\partial S}{\partial T} \right) _p
\end{equation}
相减
\begin{equation}
    C_p-C_V=T\left( \frac{\partial S}{\partial T} \right) _p-T\left( \frac{\partial S}{\partial T} \right) _V
\end{equation}

根据附录的公式

根据(这里还要再完善)
\begin{equation}
    S=S(T,V)=S(T,V(T,p))=S(T,p)
\end{equation}
得到
\begin{equation}
    \left( \frac{\partial S}{\partial T} \right) _p=\left( \frac{\partial S}{\partial T} \right) _V+\left( \frac{\partial S}{\partial V} \right) _T\left( \frac{\partial V}{\partial T} \right) _p
\end{equation}
代入得到
\begin{equation}
    \begin{aligned}
        C_p-C_V&=T\left[ \left( \frac{\partial S}{\partial T} \right) _V+\left( \frac{\partial S}{\partial V} \right) _T\left( \frac{\partial V}{\partial T} \right) _p \right] -T\left( \frac{\partial S}{\partial T} \right) _V
\\
&=T\left( \frac{\partial S}{\partial V} \right) _T\left( \frac{\partial V}{\partial T} \right) _p
    \end{aligned}
\end{equation}
利用麦克斯韦关系
\begin{equation}
    \left( \frac{\partial S}{\partial V} \right) _T=\left( \frac{\partial p}{\partial T} \right) _V
\end{equation}
得到两热容之差与物态方程的关系
\begin{equation}
    C_p-C_V=T\left( \frac{\partial p}{\partial T} \right) _V\left( \frac{\partial V}{\partial T} \right) _p
\end{equation}


\subsection{推导}

\begin{equation}
    pV=nRT
\end{equation}
写出
\begin{equation}
    \begin{aligned}
        V&=\frac{nRT}{p}
\\
p&=\frac{nRT}{V}
    \end{aligned}
\end{equation}
求偏导
\begin{equation}
    \begin{aligned}
        \left( \frac{\partial p}{\partial T} \right) _V&=\frac{nR}{V}
\\
\left( \frac{\partial V}{\partial T} \right) _p&=\frac{nR}{p}
    \end{aligned}
\end{equation}
代入
\begin{equation}
    C_p-C_V=T\left( \frac{\partial p}{\partial T} \right) _V\left( \frac{\partial V}{\partial T} \right) _p
\end{equation}
得到
\begin{equation}
    \begin{aligned}
        C_p-C_V&=T\frac{nR}{V}\cdot \frac{nR}{p}
\\
&=\frac{n^2R^2T}{pV}
\\
&=\frac{n^2R^2T}{nRT}
\\
&=nR
    \end{aligned}
\end{equation}






