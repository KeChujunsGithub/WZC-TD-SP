\section{习题4}

\newpage
\subsection{4-1}
若将U看作独立变量$T,V,n_1,\cdots,n_k$的函数,试证明: (a) $U = \sum_i n_i \frac{\partial U}{\partial n_i} + V \frac{\partial U}{\partial V}$; (b) $u_i = \frac{\partial U}{\partial n_i} + v_i \frac{\partial U}{\partial V}$

\newpage
\subsection{4-2}
证明 $\mu_i (T,p,n_1,\cdots,n_k)$ 是$n_1,\cdots,n_k$的零次齐函数 \[ \sum_i n_i \left( \frac{\partial \mu_i}{\partial n_i} \right) = 0 \]

\newpage
\subsection{4-3}
二元理想溶液具有下列形式的化学势:
$$ \mu_1 = g_1 (T,p) + RT \ln x_1, $$
$$ \mu_2 = g_2 (T,p) + RT \ln x_2, $$
其中 $g_i (T,p)$ 为纯i组元的化学势,$x_i$ 是溶液中i组元的摩尔分数。当物质的量分别为 $n_1, n_2$ 的两种纯液体在等温等压下合成理想溶液时,试证明混合前后
(a) 吉布斯函数的变化为
$$ \Delta G = RT(n_1 \ln x_1 + n_2 \ln x_2). $$
(b) 体积不变,即 $\Delta V = 0.$
(c) 熵变 $\Delta S = -R(n_1 \ln x_1 + n_2 \ln x_2).$
(d) 焓变 $\Delta H = 0$,因而没有混合热。
(e) 内能变化为多少?

\newpage
\subsection{4-4}
理想溶液中各组元的化学势为
$$ \mu_i = g_i (T,p) + RT \ln x_i. $$
(a) 假设溶质是非挥发性的。试证明,当溶液与溶剂的蒸气达到平衡时,相平衡条件为
$$ g'_i = g_1 + RT \ln (1-x), $$
其中 $g'_i$ 是蒸气的摩尔吉布斯函数,$g_1$ 是纯溶剂的摩尔吉布斯函数,$x$ 是溶质在溶液中的摩尔分数。
(b) 求证:在一定温度下,溶剂的饱和蒸气压随溶质浓度的变化率为
$$ \left( \frac{\partial p}{\partial x} \right)_T = \frac{p}{1-x}. $$
(c) 将上述积分,得
$$ p_x = p_0 (1-x), $$
其中 $p_0$ 是该温度下纯溶剂的饱和蒸气压,$p_x$ 是溶质浓度为 $x$ 时的饱和蒸气压。上式表明,溶剂饱和蒸气压的降低与溶质的摩尔分数成正比。该公式称为拉乌定律。

\newpage
\subsection{4-5}
承 4-4 题。
(a) 试证明,在一定压强下溶剂沸点随溶质浓度的变化率为 $$ \left( \frac{\partial T}{\partial x} \right)_p = \frac{RT^2}{L(1-x)} $$ ,其中 $L$ 为纯溶剂的汽化热。
(b) 假设 $x \ll 1$。试证明,溶液沸点升高与溶质在溶液中的浓度成正比,即 $$ \Delta T = \frac{RT^2}{L}x $$。

\newpage
\subsection{4-6}
如图4-1所示,开口玻璃管底端有半透膜将管中糖的水溶液与容器内的水隔开。半透膜只让水透过,不让糖透过。实验发现,糖水溶液的液面比容器内的水面上升一个高度 $h$,表明糖水溶液的压强 $p$ 与水的压强 $p_0$ 之差为
$$ p-p_0 = pgh. $$
这一压强差称为渗透压。试证明,糖水与水达到平衡时有
$$ g_1(T,p) + RT\ln(1-x) = g_1(T,p_0), $$
其中 $g_1$ 是纯水的摩尔吉布斯函数,$x$ 是糖水中糖的摩尔分数,
$$ x = \frac{n_2}{n_1+n_2} \approx \frac{n_2}{n_1} << 1(n_1,n_2) $$
分别是糖水中水和糖的物质的量)。试据此证明
$$ p-p_0 = \frac{n_2RT}{V}, $$
$V$ 是糖水溶液的体积。

\newpage
\subsection{4-7}
实验测得碳燃烧为二氧化碳和一氧化碳燃烧为二氧化碳的燃烧热 $Q=-\Delta H$,其数值分别如下:
$$ CO_2-C-O_2=0, \quad \Delta H_1=-3.951\ 8\times10^5\ J; $$
$$ CO_2-CO-\frac{1}{2}O_2=0, \quad \Delta H_2=-2.828\ 8\times10^5\ J. $$
试根据赫斯定律计算碳燃烧为一氧化碳的燃烧热。

\newpage
\subsection{4-8}
绝热容器中有隔板隔开,两边分别装有物质的量为$n_1$和$n_2$的理想气体,温度同为$T$,压强分别为$p_1$和$p_2$。今将隔板抽去,
(a) 试求气体混合后的压强。
(b) 如果两种气体是不同的,计算混合后的熵增加值。
(c) 如果两种气体是相同的,计算混合后的熵增加值。

\newpage
\subsection{4-9}
隔板将容器分为两半,各装有1 mol的理想气体A和B。它们的构成原子是相同的,不同仅在于A气体的原子核处在基态,而B气体的原子核处在激发态。已知核激发态的寿命远大于抽去隔板后气体在容器内的扩散时间。令容器与热源接触,保持恒定的温度。
(a) 如果使B气体的原子核激发后,马上抽去隔板,求扩散完成后气体的熵增加值。
(b) 如果使B气体的原子核激发后,经过远大于激发态寿命的时间再抽去隔板,求气体的熵增加值。

\newpage
\subsection{4-10}
试证明,在NH$_3$分解为N$_2$和H$_2$的反应
$$ \frac{1}{2} N_2 + \frac{3}{2} H_2 - NH_3 = 0 $$
中,平衡常量可表示为
$$ K_p = \sqrt{\frac{27}{4}} \times \frac{\varepsilon^2}{1-\varepsilon^{2}}p, $$
其中$\varepsilon$是分解度。如果将反应方程写作
$$ N_2 + 3H_2 - 2NH_3 = 0, $$
平衡常量为何?

\newpage
\subsection{4-11}
物质的量为 $n_0 \nu_1$ 的气体 A$_1$ 和物质的量为 $n_0 \nu_2$ 的气体 A$_2$ 的混合物在温度 $T$ 和压强 $p$ 下体积为 $V_0$,当发生化学变化
$$ \nu_3 A_3 + \nu_4 A_4 - \nu_1 A_1 - \nu_2 A_2 = 0, $$
并在同样的温度和压强下达到平衡时,其体积为 $\widetilde{V}$。证明反应度 $\varepsilon$ 为
$$ \varepsilon = \frac{\widetilde{V} - V_0}{V_0} \cdot \frac{\nu_1 + \nu_2}{\nu_3 + \nu_4 - \nu_1 - \nu_2}. $$

\newpage
\subsection{4-12}
试根据热力学第三定律证明,在 $T \to 0$ 时,表面张力系数与温度无关,即
$$ \frac{d\sigma}{dT} \to 0. $$

\newpage
\subsection{4-13}
设在压强 $p$ 下,某物质的熔点为 $T_0$,相变潜热为 $L$,固相的定压热容为 $C_p$,液相的定压热容为 $C'_p$。试求液相的绝对熵的表达式。

\newpage
\subsection{4-14}
锡可以形成白锡(正方晶系)和灰锡(立方晶系)两种不同的结晶状态。常压下相变温度 $T_0 = 292 \, K$。$T_0$ 以上白锡是稳定的,$T_0$ 以下灰锡是稳定的。如果在 $T_0$ 以上将白锡迅速冷却到 $T_0$ 以下,白锡将被冻结在亚稳态。已知相变潜热 $L = 2242 \, J \cdot mol^{-1}$。由热容量的测量数据知,对于灰锡
$$ \int_0^{T_0} \frac{C_g(T)}{T} dT = 44.12 \, J \cdot mol^{-1} \cdot K^{-1}, $$
对于白锡
$$ \int_0^{T_0} \frac{C_w(T)}{T} dT = 51.44 \, J \cdot mol^{-1} \cdot K^{-1}. $$
试验证能斯特定理对于亚稳态白锡的适用性。

\newpage
\subsection{4-15}
试根据热力学第三定律讨论图4-2(a),(b)两图中哪一个图是正确的? 图上画出的是顺磁性固体在 $\mathcal{H}=0$ 和 $\mathcal{H}=\mathcal{H}_i$ 时的 $S-T$ 曲线。


