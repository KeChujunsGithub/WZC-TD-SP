\section{6.7}


\newpage
\subsection{推导:玻色-爱因斯坦分布}
在玻色-爱因斯坦统计中,由费玻色-爱因斯坦系统的微观状态数推导玻色-爱因斯坦分布

1.玻色-爱因斯坦系统的微观状态数
\begin{equation}
    \Omega _{\mathrm{B}.\mathrm{E}.}=\prod_l{\frac{\left( \omega _l+a_l-1 \right) !}{a_l!\left( \omega _l-1 \right) !}}
\end{equation}
取对数
\begin{equation}
    \begin{aligned}
        \ln \Omega &=\ln \prod_l{\frac{\left( \omega _l+a_l-1 \right) !}{a_l!\left( \omega _l-1 \right) !}}
\\
&=\sum_l{\ln \frac{\left( \omega _l+a_l-1 \right) !}{a_l!\left( \omega _l-1 \right) !}}
\\
&=\sum_l{\left[ \ln \left( \omega _l+a_l-1 \right) !-\ln a_l!-\ln \left( \omega _l-1 \right) ! \right]}
    \end{aligned}
\end{equation}
2.假设$a_l\gg 1,\omega _l\gg 1$,因而$\omega _l+a_l-1\approx \omega _l+a_l,\omega _l-1\approx \omega _l$,
利用近似公式
\begin{equation}
    \ln m!=m\left( \ln m-1 \right) 
\end{equation}
写出
\begin{equation}
    \begin{aligned}
        \ln \left( \omega _l+a_l-1 \right) !&=\left( \omega _l+a_l-1 \right) \left[ \ln \left( \omega _l+a_l-1 \right) -1 \right] \approx \left( \omega _l+a_l \right) \left[ \ln \left( \omega _l+a_l \right) -1 \right] 
\\
\ln a_l!&=a_l\left( \ln a_l-1 \right) 
\\
\ln \left( \omega _l-1 \right) !&=\left( \omega _l-1 \right) \left[ \ln \left( \omega _l-1 \right) -1 \right] \approx \omega _l\left( \ln \omega _l-1 \right) 
    \end{aligned}
\end{equation}
得到
\begin{equation}
    \begin{aligned}
        \ln \Omega &=\sum_l{\left[ \left( \omega _l+a_l \right) \left[ \ln \left( \omega _l+a_l \right) -1 \right] -a_l\left( \ln a_l-1 \right) -\omega _l\left( \ln \omega _l-1 \right) \right]}
\\
&=\sum_l{\left[ \left( \omega _l+a_l \right) \ln \left( \omega _l+a_l \right) -\omega _l-a_l-a_l\ln a_l+a_l-\omega _l\ln \omega _l+\omega _l \right]}
\\
&=\sum_l{\left[ \left( \omega _l+a_l \right) \ln \left( \omega _l+a_l \right) -a_l\ln a_l-\omega _l\ln \omega _l \right]}
    \end{aligned}
\end{equation}
3.变分
\begin{equation}
    \begin{aligned}
        \delta \ln \Omega &=\delta \left\{ \sum_l{\left[ \left( \omega _l+a_l \right) \ln \left( \omega _l+a_l \right) -a_l\ln a_l-\omega _l\ln \omega _l \right]} \right\} 
\\
&=\sum_l{\delta \left[ \left( \omega _l+a_l \right) \ln \left( \omega _l+a_l \right) -a_l\ln a_l-\omega _l\ln \omega _l \right]}
\\
&=\sum_l{\left\{ \delta \left[ \left( \omega _l+a_l \right) \ln \left( \omega _l+a_l \right) \right] -\delta \left( a_l\ln a_l \right) -\delta \left( \omega _l\ln \omega _l \right) \right\}}
    \end{aligned}
\end{equation}
其中
\begin{equation}
    \begin{aligned}
        \delta \left[ \left( \omega _l+a_l \right) \ln \left( \omega _l+a_l \right) \right] &=\ln \left( \omega _l+a_l \right) \delta \left( \omega _l+a_l \right) +\left( \omega _l+a_l \right) \delta \ln \left( \omega _l+a_l \right) 
\\
&=\ln \left( \omega _l+a_l \right) \delta a_l+\left( \omega _l+a_l \right) \frac{\delta a_l}{\omega _l+a_l}
\\
&=\ln \left( \omega _l+a_l \right) \delta a_l+\delta a_l
    \end{aligned}
\end{equation}
且
\begin{equation}
    \begin{aligned}
        \delta \left( a_l\ln a_l \right) &=\ln a_l\delta a_l+a_l\delta \ln a_l
\\
&=\ln a_l\delta a_l+a_l\frac{\delta a_l}{a_l}
\\
&=\ln a_l\delta a_l+\delta a_l
    \end{aligned}
\end{equation}
且
\begin{equation}
    \delta \left( \omega _l\ln \omega _l \right) =0
\end{equation}
得到
\begin{equation}
    \begin{aligned}
        \delta \ln \Omega &=\sum_l{\left[ \ln \left( \omega _l+a_l \right) \delta a_l+\delta a_l-\ln a_l\delta a_l-\delta a_l-0 \right]}
\\
&=\sum_l{\left[ \ln \left( \omega _l+a_l \right) \delta a_l-\ln a_l\delta a_l \right]}
\\
&=\sum_l{\left[ \ln \left( \omega _l+a_l \right) -\ln a_l \right] \delta a_l}
    \end{aligned}
\end{equation}
根据极大值的分布的条件
\begin{equation}
    \delta \ln \Omega =0
\end{equation}
得到
\begin{equation}
    \sum_l{\left[ \ln \left( \omega _l+a_l \right) -\ln a_l \right] \delta a_l}=0
\end{equation}

4.分布不是完全独立的,粒子数守恒和能量守恒
\begin{equation}
    N=\sum_l{a_l},E=\sum_l{a_l\varepsilon _l}
\end{equation}
不是完全独立的,满足
\begin{equation}
    \delta N=\sum_l{\delta a_l}=0,\delta E=\sum_l{\varepsilon _l\delta a_l}=0
\end{equation}
两边乘以拉格朗日乘子
\begin{equation}
    \alpha \delta N=\sum_l{\alpha \delta a_l}=0,\beta \delta E=\sum_l{\beta \varepsilon _l\delta a_l}=0
\end{equation}
对于,减去
\begin{equation}
    \delta \ln \Omega -\alpha \delta N-\beta \delta E=0
\end{equation}
即
\begin{equation}
    \begin{aligned}
        \sum_l{\left[ \ln \left( \omega _l+a_l \right) -\ln a_l \right] \delta a_l}-\sum_l{\alpha \delta a_l}-\sum_l{\beta \varepsilon _l\delta a_l}&=0
\\
\sum_l{\left[ \ln \left( \omega _l+a_l \right) -\ln a_l-\alpha -\beta \varepsilon _l \right] \delta a_l}&=0
    \end{aligned}
\end{equation}

根据拉格朗日乘子法原理
\begin{equation}
    \ln \left( \omega _l+a_l \right) -\ln a_l-\alpha -\beta \varepsilon _l=0
\end{equation}
计算
\begin{equation}
    \begin{aligned}
        \ln \frac{\omega _l+a_l}{a_l}=\alpha +\beta \varepsilon _l
\\
\frac{\omega _l+a_l}{a_l}=e^{\alpha +\beta \varepsilon _l}
\\
\frac{\omega _l}{a_l}+1=e^{\alpha +\beta \varepsilon _l}
\\
\frac{\omega _l}{a_l}=e^{\alpha +\beta \varepsilon _l}-1
    \end{aligned}
\end{equation}
得到
\begin{equation}
    a_l=\frac{\omega _l}{e^{\alpha +\beta \varepsilon _l}-1}
\end{equation}

%%%%%%%%%%%%%%%%%%%%%%%%%%%%%%%%%%%%%%%%%%%%%%%%%%%%%%%%
\newpage
\subsection{推导:费米-狄拉克分布}
在费米-狄拉克统计中,由费米-狄拉克系统的微观状态数推导费米-狄拉克分布

1.费米-狄拉克系统的微观状态数
\begin{equation}
    \Omega _{\mathrm{F}.\mathrm{D}.}=\prod_l{\frac{\omega _l!}{a_l!\left( \omega _l-a_l \right) !}}
\end{equation}
取对数
\begin{equation}
    \begin{aligned}
        \ln \Omega &=\ln \prod_l{\frac{\omega _l!}{a_l!\left( \omega _l-a_l \right) !}}
\\
&=\sum_l{\ln \frac{\omega _l!}{a_l!\left( \omega _l-a_l \right) !}}
\\
&=\sum_l{\left[ \ln \omega _l!-\ln a_l!-\ln \left( \omega _l-a_l \right) ! \right]}
    \end{aligned}
\end{equation}
2.假设$a_l\gg 1,\omega _l\gg 1,\omega _l-a_l\gg 1$
利用近似公式
\begin{equation}
    \ln m!=m\left( \ln m-1 \right) 
\end{equation}
写出
\begin{equation}
    \begin{aligned}
        \ln \omega _l!&=\omega _l\left( \ln \omega _l-1 \right) 
\\
\ln a_l!&=a_l\left( \ln a_l-1 \right) 
\\
\ln \left( \omega _l-a_l \right) !&=\left( \omega _l-a_l \right) \left[ \ln \left( \omega _l-a_l \right) -1 \right] 
    \end{aligned}
\end{equation}
得到
\begin{equation}
    \begin{aligned}
        \ln \Omega &=\sum_l{\left[ \omega _l\left( \ln \omega _l-1 \right) -a_l\left( \ln a_l-1 \right) -\left( \omega _l-a_l \right) \left( \ln \left( \omega _l-a_l \right) -1 \right) \right]}
\\
&=\sum_l{\left[ \omega _l\ln \omega _l-\omega _l-a_l\ln a_l+a_l-\left( \omega _l-a_l \right) \ln \left( \omega _l-a_l \right) -a_l+\omega _l \right]}
\\
&=\sum_l{\left[ \omega _l\ln \omega _l-a_l\ln a_l-\left( \omega _l-a_l \right) \ln \left( \omega _l-a_l \right) \right]}
    \end{aligned}
\end{equation}
3.变分
\begin{equation}
    \begin{aligned}
        \delta \ln \Omega &=\delta \left\{ \sum_l{\left[ \omega _l\ln \omega _l-a_l\ln a_l-\left( \omega _l-a_l \right) \ln \left( \omega _l-a_l \right) \right]} \right\} 
\\
&=\sum_l{\delta \left[ \omega _l\ln \omega _l-a_l\ln a_l-\left( \omega _l-a_l \right) \ln \left( \omega _l-a_l \right) \right]}
\\
&=\sum_l{\left\{ \delta \left( \omega _l\ln \omega _l \right) -\delta \left( a_l\ln a_l \right) -\delta \left[ \left( \omega _l-a_l \right) \ln \left( \omega _l-a_l \right) \right] \right\}}
    \end{aligned}
\end{equation}
其中
\begin{equation}
    \delta \left( \omega _l\ln \omega _l \right) =0
\end{equation}
且
\begin{equation}
    \begin{aligned}
        \delta \left( a_l\ln a_l \right) &=\ln a_l\delta a_l+a_l\delta \ln a_l
\\
&=\ln a_l\delta a_l+a_l\frac{\delta a_l}{a_l}
\\
&=\ln a_l\delta a_l+\delta a_l
    \end{aligned}
\end{equation}
且
\begin{equation}
    \begin{aligned}
        \delta \left[ \left( \omega _l-a_l \right) \ln \left( \omega _l-a_l \right) \right] &=\ln \left( \omega _l-a_l \right) \delta \left( \omega _l-a_l \right) +\left( \omega _l-a_l \right) \delta \ln \left( \omega _l-a_l \right) 
\\
&=\ln \left( \omega _l-a_l \right) \delta a_l+\left( \omega _l-a_l \right) \small{\frac{\delta a_l}{\omega _l-a_l}}
\\
&=\ln \left( \omega _l-a_l \right) \delta a_l+\delta a_l
    \end{aligned}
\end{equation}
得到
\begin{equation}
    \begin{aligned}
        \delta \ln \Omega &=\sum_l{\left[ 0-\ln a_l\delta a_l-\delta a_l+\ln \left( \omega _l-a_l \right) \delta a_l+\delta a_l \right]}
\\
&=\sum_l{\left[ -\ln a_l\delta a_l+\ln \left( \omega _l-a_l \right) \delta a_l \right]}
\\
&=\sum_l{\left[ \ln \left( \omega _l-a_l \right) -\ln a_l \right] \delta a_l}
    \end{aligned}
\end{equation}

4.根据极大值的分布的条件
\begin{equation}
    \delta \ln \Omega =0
\end{equation}
得到
\begin{equation}
    \sum_l{\left[ \ln \left( \omega _l-a_l \right) -\ln a_l \right] \delta a_l}=0
\end{equation}
4.分布不是完全独立的,粒子数守恒和能量守恒
\begin{equation}
    N=\sum_l{a_l},E=\sum_l{a_l\varepsilon _l}
\end{equation}
不是完全独立的,满足
\begin{equation}
    \delta N=\sum_l{\delta a_l}=0,\delta E=\sum_l{\varepsilon _l\delta a_l}=0
\end{equation}
两边乘以拉格朗日乘子
\begin{equation}
    \alpha \delta N=\sum_l{\alpha \delta a_l}=0,\beta \delta E=\sum_l{\beta \varepsilon _l\delta a_l}=0
\end{equation}
减去
\begin{equation}
    \delta \ln \Omega -\alpha \delta N-\beta \delta E=0
\end{equation}
即
\begin{equation}
    \begin{aligned}
        \sum_l{\left[ \ln \left( \omega _l-a_l \right) -\ln a_l \right] \delta a_l}-\sum_l{\alpha \delta a_l}-\sum_l{\beta \varepsilon _l\delta a_l}&=0
\\
\sum_l{\left[ \ln \left( \omega _l-a_l \right) -\ln a_l-\alpha -\beta \varepsilon _l \right] \delta a_l}&=0
    \end{aligned}
\end{equation}

根据拉格朗日乘子法原理
\begin{equation}
    \mathrm{ln}\left( \omega _l-a_l \right) -\ln a_l-\alpha -\beta \varepsilon _l=0
\end{equation}
计算
\begin{equation}
    \begin{aligned}
        \ln \frac{\omega _l-a_l}{a_l}=\alpha +\beta \varepsilon _l
\\
\frac{\omega _l-a_l}{a_l}=e^{\alpha +\beta \varepsilon _l}
\\
\frac{\omega _l}{a_l}-1=e^{\alpha +\beta \varepsilon _l}
\\
\frac{\omega _l}{a_l}=e^{\alpha +\beta \varepsilon _l}+1
    \end{aligned}
\end{equation}
得到费米-狄拉克分布
\begin{equation}
    a_l=\frac{\omega _l}{e^{\alpha +\beta \varepsilon _l}+1}
\end{equation}
