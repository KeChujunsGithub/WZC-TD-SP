\section{习题6}

\newpage
\subsection{6-1}
试根据式(6.2.13)证明:在体积$V$内,在$\varepsilon$到$\varepsilon+d\varepsilon$的能量范围内,三维自由粒子的量子态数为
$$D(\varepsilon)d\varepsilon = \frac{2\pi V}{h^3}(2m)^{\frac{3}{2}}\varepsilon^{\frac{1}{2}}d\varepsilon.$$

\newpage
\subsection{6-2}
试证明,对于二维自由粒子,在长度$L$内,在$\varepsilon$到$\varepsilon+d\varepsilon$的能量范围内,量子态数为
$$D(\varepsilon)d\varepsilon = \frac{2L}{h}\left(\frac{m}{2\varepsilon}\right)^{\frac{1}{2}}d\varepsilon.$$

\newpage
\subsection{6-3}
试证明,对于二维自由粒子,在面积$L^2$内,在$\varepsilon$到$\varepsilon + d\varepsilon$的能量范围内,量子态数为
$$D(\varepsilon) d\varepsilon = \frac{2\pi L^2}{h^2} m d\varepsilon.$$

\newpage
\subsection{6-4}
在极端相对论情形下,粒子的能量动量关系为
$$\varepsilon = cp.$$
试求在体积$V$内,在$\varepsilon$到$\varepsilon + d\varepsilon$的能量范围内三维粒子的量子态数。

\newpage
\subsection{6-5}
设系统含有两种粒子,其粒子数分别为$N$和$N'$。粒子间的相互作用很弱,可以看作是近独立的。假设粒子可以分辨,处在一个个体量子态的粒子数不受限制。试证明,在平衡状态下两种粒子的最概然分布分别为
$$a_l = \omega_l e^{-\alpha - \beta \varepsilon_l}$$
和
$$a_l' = \omega_l' e^{-\alpha' - \beta \varepsilon_l'},$$
其中$\varepsilon_l$和$\varepsilon_l'$是两种粒子的能级,$\omega_l$和$\omega_l'$是能级的简并度。

\newpage
\subsection{6-6}
同上题,如果粒子是玻色子或费米子,结果如何?