\section{习题1}


\newpage
\subsection{1-1}
试求理想气体的体胀系数 $\alpha$,压强系数 $\beta$ 和等温压缩系数 $\kappa_T$。

\subsubsection{解答}
根据理想气体状态方程
\begin{equation}
    pV=nRT
\end{equation}
写出
\begin{equation}
    \begin{aligned}
        V&=\frac{nRT}{p}
\\
p&=\frac{nRT}{V}
    \end{aligned}
\end{equation}
求导得到
\begin{equation}
    \begin{aligned}
        \biggl( \frac{\partial V}{\partial T} \biggr) _p&=\frac{nR}{p}
\\
\left( \frac{\partial V}{\partial p} \right) _T&=-\frac{nRT}{p^2}
\\
\biggl( \frac{\partial p}{\partial T} \biggr) _V&=\frac{nR}{V}
    \end{aligned}
\end{equation}
代入体胀系数
\begin{equation}
    \alpha =\frac{1}{V}\biggl( \frac{\partial V}{\partial T} \biggr) _p
\end{equation}
得到
\begin{equation}
    \begin{aligned}
        \alpha &=\frac{1}{V}\frac{nR}{p}
\\
&=\frac{nR}{pV}
\\
&=\frac{nR}{nRT}
\\
&=\frac{1}{T}
    \end{aligned}
\end{equation}
代入压强系数
\begin{equation}
    \beta =\frac{1}{p}\biggl( \frac{\partial p}{\partial T} \biggr) _V
\end{equation}
得到
\begin{equation}
    \begin{aligned}
        \beta &=\frac{1}{p}\frac{nR}{V}
\\
&=\frac{nR}{pV}
\\
&=\frac{nR}{nRT}
\\
&=\frac{1}{T}
    \end{aligned}
\end{equation}
代入等温压缩系数
\begin{equation}
    \kappa _T=-\frac{1}{V}\left( \frac{\partial V}{\partial p} \right) _T
\end{equation}
得到
\begin{equation}
    \begin{aligned}
        \kappa _T&=-\frac{1}{V}\left( -\frac{nRT}{p^2} \right) 
\\
&=\frac{nRT}{p^2V}
\\
&=\frac{pV}{p^2V}
\\
&=\frac{1}{p}
    \end{aligned}
\end{equation}
综上,理想气体的体胀系数 $\alpha$,压强系数 $\beta$ 和等温压缩系数 $\kappa_T$分别为
\begin{equation}
    \begin{aligned}
        \alpha =\frac{1}{T}
\\
\beta =\frac{1}{T}
\\
\kappa _T=\frac{1}{p}
    \end{aligned}
\end{equation}









\newpage
\subsection{1-2}
试证明任何一种具有两个独立参量 $T, p$ 的物质,其物态方程可由实验测得的体胀系数 $\alpha$ 及等温压缩系数 $\kappa_T$,根据下述积分求得:
$$\ln V = \int (\alpha dT - \kappa_T dp).$$
如果 $\alpha = \frac{1}{T}, \kappa_T = \frac{1}{p}$,试求物态方程。

\subsubsection{解答}
具有
\begin{equation}
    V=V\left( T,p \right) 
\end{equation}
求全微分
\begin{equation}
    \mathrm{d}V=\left( \frac{\partial V}{\partial T} \right) _p\mathrm{d}T+\left( \frac{\partial V}{\partial p} \right) _T\mathrm{d}p
\end{equation}
两边同时乘以
\begin{equation}
    \frac{\mathrm{d}V}{V}=\frac{1}{V}\left( \frac{\partial V}{\partial T} \right) _p\mathrm{d}T+\frac{1}{V}\left( \frac{\partial V}{\partial p} \right) _T\mathrm{d}p
\end{equation}
根据体胀系数 $\alpha$和等温压缩系数 $\kappa_T$
\begin{equation}
    \alpha =\frac{1}{V}\left( \frac{\partial V}{\partial T} \right) _p,\kappa _T=-\frac{1}{V}\left( \frac{\partial V}{\partial p} \right) _T
\end{equation}
得到
\begin{equation}
    \frac{\mathrm{d}V}{V}=\alpha \mathrm{d}T-\kappa _T\mathrm{d}p
\end{equation}
两边同时不定积分
\begin{equation}
    \int{\frac{\mathrm{d}V}{V}}=\int{\left( \alpha \mathrm{d}T-\kappa _T\mathrm{d}p \right)}
\end{equation}
得到
\begin{equation}
    \ln V=\int{\left( \alpha \mathrm{d}T-\kappa _T\mathrm{d}p \right)}
\end{equation}

\subsubsection{解答第二小问}
对于
\begin{equation}
    \frac{\mathrm{d}V}{V}=\alpha \mathrm{d}T-\kappa _T\mathrm{d}p
\end{equation}
根据习题1.1理想气体的体胀系数 $\alpha$和等温压缩系数 $\kappa_T$
\begin{equation}
    \alpha =\frac{1}{T},\kappa _T=\frac{1}{p}
\end{equation}
得到
\begin{equation}
    \frac{dV}{V}=\frac{1}{T}\mathrm{d}T-\frac{1}{p}\mathrm{d}p
\end{equation}
两边同时积分

得到




\newpage
\subsection{1-3}
简单固体和液体的体胀系数 $\alpha$ 和等温压缩系数 $\kappa_T$ 数值都很小,在一定温度范围内可以把 $\alpha$ 和 $\kappa_T$ 看作常量。试证明简单固体和液体的物态方程可近似表示为 [式 (1.3.14)]
$$V(T, p) = V_0 (T_0, 0) \left[ 1 + \alpha (T - T_0) - \kappa_T p \right].$$

\newpage
\subsection{1-4}
在0℃和1 atm下,测得一铜块的体胀系数和等温压缩系数分别为 $\alpha = 4.85 \times 10^{-5} K^{-1}$ 和 $\kappa_T = 7.8 \times 10^{-7} atm^{-1}$。$\alpha$ 和 $\kappa_T$ 可近似看作常量。今使铜块加热至10℃。问:
(a) 压强要增加多少 atm 才能使铜块的体积维持不变?
(b) 若压强增加100 atm,铜块的体积改变多少?

\newpage
\subsection{1-5}
描述金属丝的几何参量是长度 $L$,力学参量是张力 $\mathcal{F}$,物态方程是
$$f(\mathcal{F}, L, T) = 0.$$
实验通常在1 atm下进行,其体积变化可以忽略。
线胀系数定义为
$$\alpha = \frac{1}{L} \left( \frac{\partial L}{\partial T} \right)_{\mathcal{F}}.$$
等温弹性模量定义为
$$E = \frac{L}{A} \left( \frac{\partial \mathcal{F}}{\partial L} \right)_T,$$
其中 $A$ 是金属丝的截面积。一般来说,$\alpha$ 和 $E$ 是 $T$ 的函数,对 $\mathcal{F}$ 仅有微弱的依赖关系,如果温度变化范围不大,可以看作常量。假设金属丝两端固定,试证明,当温度由 $T_1$ 降至 $T_2$ 时,其张力将增加
$$\Delta \mathcal{F} = - E A \alpha (T_2 - T_1).$$

\newpage
\subsection{1-6}
一理想弹性丝的物态方程为
$$\mathcal{F} = bT \left( \frac{L}{L_0} - \frac{L_0^2}{L^2} \right),$$
其中 $L$ 是长度,$L_0$ 是张力 $\mathcal{F}$ 为零时的 $L$ 值,它只是温度 $T$ 的函数,$b$ 是常量。试证明:
(a) 等温弹性模量为
$$E = \frac{bT}{A} \left( \frac{L}{L_0} + \frac{2L_0^2}{L^2} \right).$$
式中 $A$ 是弹性丝的截面积。在张力为零时,$E_0 = \frac{3bT}{A}$。
(b) 线胀系数 $\alpha$ 为
$$\alpha = \alpha_0 - \frac{1}{T} \frac{\frac{L^3}{L_0^3} - 1}{\frac{L^3}{L_0^3} + 2},$$
其中 $\alpha_0 = \frac{1}{L_0} \frac{dL_0}{dT}$。

\newpage
\subsection{1-7}
在 0 ℃ 和 1 atm 下,空气的密度为 $1.29 \mathrm{kg \cdot m^{-3}}$。空气的比定压热容 $c_p = 0.996 \times 10^3 \mathrm{J \cdot kg^{-1} \cdot K^{-1}}$,$\gamma = 1.41$。今有 $27 \mathrm{m^3}$ 的空气,试计算:
(a) 若维持体积不变,将空气由 0 ℃ 加热至 20 ℃ 所需的热量。
(b) 若维持压强不变,将空气由 0 ℃ 加热至 20 ℃ 所需的热量。
(c) 若容器有裂缝,外界压强为 1 atm,使空气由 0 ℃ 缓慢地加热至 20 ℃ 所需的热量。

\newpage
\subsection{1-8}
抽成真空的小匣带有活门,打开活门让气体冲入。当压强达到外界压强 $p_0$ 时将活门关上。试证明:小匣内的空气在没有与外界交换热量之前,它的内能 $U$ 与原来在大气中的内能 $U_0$ 之差为 $U - U_0 = p_0 V_0$,其中 $V_0$ 是它原来在大气中的体积。若气体是理想气体,求它的温度和体积。

\newpage
\subsection{1-9}
满足 $pV^n = C$(常量)的过程称为多方过程,其中常数 $n$ 称为多方指数。试证明:理想气体在多方过程中的热容 $C_n$ 为
$$C_n = \frac{n-\gamma}{n-1}C_V.$$

\newpage
\subsection{1-10}
试证明,理想气体在某一过程中的热容 $C_n$ 如果是常量,该过程一定是多方过程。多方指数 $n=\frac{C_n-C_p}{C_n-C_V}$。假设气体的定压热容和定容热容是常量。

\newpage
\subsection{1-11}
声波在气体中的传播速度为
$$a = \sqrt{\left( \frac{\partial p}{\partial \rho} \right)_s}.$$
假设气体是理想气体,其定压和定容热容是常量。试证明气体单位质量的内能 $u$ 和焓 $h$ 可由声速 $a$ 及 $\gamma$ 给出:
$$u = \frac{a^2}{\gamma(\gamma - 1)} + u_0,$$
$$h = \frac{a^2}{\gamma - 1} + h_0,$$
其中 $u_0, h_0$ 为常量。

\newpage
\subsection{1-12}
大气温度随高度降低的主要原因是在对流层中不同高度之间的空气不断发生对流。由于气压随高度而降低,空气上升时膨胀,下降时收缩。空气的导热系数很小,膨胀和收缩的过程可以认为是绝热过程。试计算大气温度随高度的变化率 $\frac{dT}{dz}$,并给出数值结果。

\newpage
\subsection{1-13}
假设理想气体的 $C_p$ 和 $C_V$ 之比 $\gamma$ 是温度的函数。试求在准静态绝热过程中 $T$ 和 $V$ 的关系。该关系式中要用到一个函数 $F(T)$,其表达式为
$$\ln F(T) = \int \frac{dT}{(\gamma - 1)T}.$$

\newpage
\subsection{1-14}
利用上题的结果证明,当 $\gamma$ 是温度的函数时,理想气体卡诺循环的效率仍为
$$\eta = 1 - \frac{T_2}{T_1}.$$

\newpage
\subsection{1-15}
试根据热力学第二定律证明两条绝热线不能相交。

\newpage
\subsection{1-16}
热泵的作用是通过一个循环过程将热量从温度较低的环境传送到温度较高的物体上去。如果以逆卡诺循环作为热泵的循环过程,热泵的效率可以定义为传送到高温物体的热量与外界所做的功的比值。试求热泵的效率。如果将功直接转化为热量而令高温物体吸收,则"效率"为何?

\newpage
\subsection{1-17}
热机在循环中与多个热源交换热量,在热机从其中吸取热量的热源中,热源的最高温度为 $T_1$。在热机向其放出热量的热源中,热源的最低温度为 $T_2$。试根据克劳修斯不等式证明,热机的效率不超过 $1 - \frac{T_2}{T_1}$。

\newpage
\subsection{1-18}
理想气体分别经等压过程和等体过程,温度由 $T_1$ 升至 $T_2$。假设 $\gamma$ 是常数,试证明前者的熵增加值为后者的 $\gamma$ 倍。

\newpage
\subsection{1-19}
温度为0℃的1kg水与温度为100℃的恒温热源接触后,水温达到100℃。试分别求水和热源的熵变以及整个系统的总熵变。欲使参与过程的整个系统的熵保持不变,应如何使水温从0℃升至100℃?已知水的比热容为 $4.18 \mathrm{J \cdot g^{-1} \cdot K^{-1}}$。

\newpage
\subsection{1-20}
10 A的电流通过一个25 $\Omega$的电阻器,历时1 s。
(a) 若电阻器保持为室温27 ℃,试求电阻器的熵增加值。
(b) 若电阻器被一绝热壳包装起来,其初温为27 ℃,电阻器的质量为10 g,比定压热容 $c_p$ 为 $0.84 \mathrm{J \cdot g^{-1} \cdot K^{-1}}$,问电阻器的熵增加值为多少?

\newpage
\subsection{1-21}
均匀杆的温度一端为 $T_1$,另一端为 $T_2$。试计算达到均匀温度 $\frac{1}{2}(T_1 + T_2)$ 后的熵增加值。

\newpage
\subsection{1-22}
一物质固态的摩尔热容为 $C_s$,液态的摩尔热容为 $C_L$。假设 $C_s$ 和 $C_L$ 都可看作常量。在某一压强下,该物质的熔点为 $T_0$,相变潜热为 $Q_0$。求在温度为 $T_1 (T_1 < T_0)$ 时,过冷液体与同温度下固体的摩尔熵差。假设过冷液体的摩尔热容亦为 $C_L$。

\newpage
\subsection{1-23}
根据熵增加原理证明热力学第二定律的开尔文表述,从单一热源吸取热量使之完全变成有用的功而不引起其他变化是不可能的。

\newpage
\subsection{1-24}
物体的初温 $T_1$ 高于热源的温度 $T_2$。有一热机在此物体与热源之间工作,直到将物体的温度降低到 $T_2$ 为止。若热机从物体吸取的热量为 $Q$,试根据熵增加原理证明,此热机所能输出的最大功为
$$W_{\text{max}} = Q - T_2 (S_1 - S_2),$$
其中 $S_1 - S_2$ 是物体的熵减少量。

\newpage
\subsection{1-25}
有两个状态相同的物体,热容为常量,初始温度同为 $T_i$。今令一制冷机在这两个物体间工作,使其中一个物体的温度降低到 $T_2$ 为止。假设物体维持在定压下,并且不发生相变。试根据熵增加原理证明,此过程所需的最小功为
$$W_{\text{min}} = C_p \left( \frac{T_i^2}{T_2} + T_2 - 2T_i \right).$$

\newpage
\subsection{1-26}
简单系统有两个独立参量。如果以 $T, S$ 为独立参量,可用纵坐标表示温度 $T$,横坐标表示熵 $S$,构成 $T-S$ 图。图中的一点与系统的一个平衡态相对应,一条曲线与一个可逆过程相对应。试在图中画出可逆卡诺循环过程的曲线,并利用 $T-S$ 图求可逆卡诺循环的效率。
