\section{习题9}


\newpage
\subsection{9-1}
试证明在微正则系统理论中熵可表示为
$$ S = -k \sum_{s} \rho_{s} \ln \rho_{s}, $$
其中 $\rho_{s} = \frac{1}{\Omega}$ 是系统处在状态 $s$ 的概率,$\Omega$ 是系统可能的微观状态数。

\newpage
\subsection{9-2}
证明在正则分布中熵可表示为
$$ S = -k \sum_{s} \rho_{s} \ln \rho_{s}, $$
其中 $\rho_{s} = \frac{1}{Z} e^{-\beta E_{s}}$ 是系统处在能量为 $E_{s}$ 的状态 $s$ 的概率。

\newpage
\subsection{9-3}
试应用正则分布求单原子分子理想气体的物态方程、内能、熵和化学势。

\newpage
\subsection{9-4}
试根据正则系综理论的涨落公式求单原子和双原子分子理想气体的能量相对涨落。

\newpage
\subsection{9-5}
体积为 $V$ 的容器内盛有 A、B 两种组元的单原子分子混合理想气体,其原子数分别为 $N_A$ 和 $N_B$,温度为 $T$。试应用正则系综理论求混合理想气体的物态方程、内能和熵。

\newpage
\subsection{9-6}
气体含 $N$ 个极端相对论粒子,粒子之间的相互作用可以忽略。假设经典极限条件得到满足,试用正则系综理论求气体的物态方程、内能、熵和化学势。

\newpage
\subsection{9-7}
试根据正则分布导出实际气体分子的速度分布。

\newpage
\subsection{9-8}
被吸附在液体表面的分子形成一种二维气体。考虑到分子间的相互作用,试证明,二维气体的物态方程可以近似为
$$ pA = NkT \left( 1 + \frac{N}{N_A} \cdot \frac{B}{A} \right), $$
其中
$$ B = \frac{N_A}{2} \int \left( e^{-\frac{\phi}{kT}} - 1 \right) 2 \pi rdr, $$
$A$ 是液面的面积,$\phi$ 是两分子的相互作用势。

\newpage
\subsection{9-9}
以 $H(q_1, \cdots, q_{3N}; p_1, \cdots, p_{3N})$ 表示由 $N$ 个经典粒子组成的系统的哈密顿量,试由正则分布证明广义能量均分定理:
$$ \overline{x_i \frac{\partial H}{\partial x_j}} = \delta_{ij} kT, $$
其中 $x_i$ 或 $x_j$ 分别是 $6N$ 个广义坐标和动量中的任意一个。由广义能量均分定理证明
$$ \sum_i \overline{q_i F_i} = -3NkT, $$
式中 $F_i$ 是作用在 $i$ 自由度的力。上式意味着,系统各自自度的坐标 $q_i$ 与对该自由度的作用力 $F_i$ 的乘积之和的统计平均值等于 $-3NkT$。克劳修斯 (Clausius) 称 $\sum_i q_i F_i$ 为位力,称上式为位力定理。

\newpage
\subsection{9-10}
将位力定理用于理想气体。导出理想气体的物态方程
$$ pV = NkT $$
和压强与气体分子平动能量 $K$ 之间的关系:
$$ p = \frac{2}{3} \frac{K}{V}. $$

\newpage
\subsection{9-11}
将位力定理用于实际气体,导出实际气体近似的物态方程
$$ pV = NkT \left( 1 + \frac{N}{N_A} \frac{B}{V} \right), $$
其中 $B$ 是第二位力系数。

\newpage
\subsection{9-12}
仿照三维固体的德拜理论,计算长度为 $L$ 的线形原子链(一维晶体)在高温和低温下的内能和热容。

\newpage
\subsection{9-13}
仿照三维固体的德拜理论,计算面积为 $L^2$ 的原子层(二维晶体)在高温和低温下的内能和热容。

\newpage
\subsection{9-14}
利用德拜频谱求固体在高温和低温下配分函数的对数 $\ln Z$,从而求内能和熵。

\newpage
\subsection{9-15}
固体中某种准粒子遵从玻色分布,具有色散关系 $\omega = Ak^2$。试证明在低温范围,这种准粒子的激发所导致的热容与 $T^{3/2}$ 成比例。铁磁体中的自旋波具有这种性质。

\newpage
\subsection{9-16}
试根据伊辛模型的平均场理论,导出弱场高温条件下顺磁性固体的物态方程——居里-外斯定律。
$$ M = \frac{C}{T - \theta} H $$

\newpage
\subsection{9-17}
用平均场近似导出非理想气体的范德瓦耳斯方程。

\newpage
\subsection{9-18}
试用巨正则分布导出单原子分子理想气体的物态方程、内能、熵和化学势。

\newpage
\subsection{9-19}
根据巨正则系综理论的涨落公式,求单原子分子和双原子分子理想气体的分子数相对涨落。

\newpage
\subsection{9-20}
证明在巨正则系综理论中熵可表示为
$$ S = -k \sum_N \sum_s \rho_{N,s} \ln \rho_{N,s}, $$
其中 $\rho_{N,s} = \frac{1}{\Xi} e^{-\alpha N - \beta E_s}$ 是系统具有 $N$ 个粒子、处在状态 $s$ 的概率。

\newpage
\subsection{9-21}
体积 $V$ 内含有 $N$ 个粒子,试用巨正则系综理论证明,在一小体积 $v$ 中有 $n$ 个粒子的概率为
$$ P_n = \frac{1}{n!} e^{-\bar{n}} (\bar{n})^n, $$
其中 $\bar{n}$ 为体积 $v$ 内的平均粒子数。上式称为泊松 (Poisson) 分布。

\newpage
\subsection{9-22}
格气模型假设原子只能取一系列分立的位置,这些位置形成一个晶格。每一格点最多为一个原子占据,即处在格点 $i$ 上的原子数 $n_i$ 可为 $0$ 或 $1$。以 $-\epsilon$ 表示处在两个近邻格点的原子的相互作用能量,系统的能量可以表示为
$$ -\epsilon \sum_{\langle i,j \rangle} n_i n_j, $$
式中 $N$ 是模型的格点数,$\sum_{\langle i,j \rangle}$ 表示对 $i,j$ 求和时只对近邻格点对求和。试写出格气模型的巨配分函数,说明它与伊辛模型的正则配分函数同构。

\newpage
\subsection{9-23}
设单原子分子理想气体与固体吸附面接触达到平衡。被吸附的分子可以在吸附面上作二维运动,其能量为 $\frac{p^2}{2m} - \varepsilon_0$,束缚能 $\varepsilon_0$ 是大于零的常量。试应用巨正则系综理论求吸附面上被吸附分子的面密度与气体温度和压强的关系。

\newpage
\subsection{9-24}
试由巨正则系综理论导出玻耳兹曼分布。

\newpage
\subsection{9-25}
试证明玻耳兹曼分布的涨落为
$$ \overline{(a_l - \bar{a}_l)^2} = \bar{a}_l. $$

\newpage
\subsection{9-26}
光子气体的 $\alpha = 0$,式 (9.12.11) 不能用。试证明,
$$ \overline{(a_l - \bar{a}_l)^2} = \frac{1}{\beta} \frac{\partial \bar{a}_l}{\partial \varepsilon_l}, $$
从而证明光子气体的涨落仍为
$$ \overline{(a_l - \bar{a}_l)^2} = \bar{a}_l (1 + \bar{a}_l). $$

\newpage
\subsection{9-27}
以 $\rho_s (s=1,2,\cdots)$ 表示系统处在状态 $s$ 的概率。$\rho_s$ 满足归一化条件
$$ \sum_{s} \rho_s = 1. $$
定义系统的熵为
$$ S=-k \sum_{s} \rho_s \ln \rho_s. $$
(a) 试证明,由式 (2) 定义的熵满足可加性要求。
(b) 如果系统的粒子数 $N$,体积 $V$ 和能量 $E$ 恒定,试证明,使式 (2) 的熵取极大值的概率分布是微正则分布。
(c) 如果系统的粒子数 $N$ 和体积 $V$ 恒定,但可与外界交换能量而能量的平均值恒定,试证明,使式 (2) 的熵取极大的概率分布是正则分布。
(d) 如果系统的体积 $V$ 恒定而可与外界交换粒子和能量,但粒子数和能量的平均值恒定,试证明,使式 (2) 的熵取极大的概率分布是巨正则分布。

