\section{习题2}

\newpage
\subsection{2-1}
温度维持为25$^\circ$C,压强在0$\sim$1000 atm之间,测得水的实验数据如下:

$$\left( \frac{\partial V}{\partial T} \right)_p = (4.5 \times 10^{-3} + 1.4 \times 10^{-6} p) \quad (\text{SI单位})$$

若在25$^\circ$C的恒温下将水从1 atm加压至1000 atm,求水的熵增加值和从外界吸收的热量。

\newpage
\subsection{2-2}
已知在体积保持不变时,一气体的压强正比于其温度。试证明在温度保持不变时,该气体的熵随体积而增加。

\newpage
\subsection{2-3}
设一物质的物态方程具有以下形式:
$$p = f(V)T,$$
试证明其内能与体积无关。

\newpage
\subsection{2-4}
求证:
(a) $\left( \frac{\partial S}{\partial p} \right)_H < 0$;
(b) $\left( \frac{\partial S}{\partial V} \right)_U > 0$.

\newpage
\subsection{2-5}
已知 $\left( \frac{\partial U}{\partial V} \right)_T = 0$,求证 $\left( \frac{\partial U}{\partial p} \right)_T = 0$.

\newpage
\subsection{2-6}
试证明一个均匀物体在准静态等压过程中熵随体积的增减取决于等压下温度随体积的增减。

\newpage
\subsection{2-7}
水的体胀系数$\alpha$在0$^\circ$C$<t<$4$^\circ$C时为负值。试证明在这温度范围内,水在绝热压缩时变冷。(其他液体和所有气体在绝热压缩时都升温)。

\newpage
\subsection{2-8}
试证明在相同的压强降落下,气体在准静态绝热膨胀中的温度降落大于在节流过程中的温度降落。

\newpage
\subsection{2-9}
实验发现,一气体的压强 $p$ 与体积 $V$ 的乘积以及内能 $U$ 都只是温度的函数,即
$$pV = f(T),$$
$$U = U(T).$$

试根据热力学理论,讨论该气体的物态方程可能具有什么形式。

\newpage
\subsection{2-10}
证明

$$\left( \frac{\partial C_V}{\partial V} \right)_T = T \left( \frac{\partial^2 p}{\partial T^2} \right)_V,$$
$$\left( \frac{\partial C_p}{\partial p} \right)_T = -T \left( \frac{\partial^2 V}{\partial T^2} \right)_p,$$

并由此导出

$$C_V = C_V^0 + T \int_{V_0}^V \left( \frac{\partial^2 p}{\partial T^2} \right)_V dV,$$
$$C_p = C_p^0 - T \int_{P_0}^P \left( \frac{\partial^2 V}{\partial T^2} \right)_p dp.$$

根据以上两式证明,理想气体的定容热容和定压热容只是温度 $T$ 的函数。

\newpage
\subsection{2-11}
证明范德瓦耳斯气体的定容热容只是温度 $T$ 的函数,与比体积无关。

\newpage
\subsection{2-12}
证明理想气体的摩尔自由能可以表示为

$$F_m = \int C_{V,m} dT + U_{m0} - T \int \frac{C_{V,m}}{T} dT - RT \ln V_m - TS_{m0}$$

$$= - T \int \frac{dT}{T^2} \int C_{V,m} dT + U_{m0} - TS_{m0} - RT \ln V_m$$

\newpage
\subsection{2-13}
求范德瓦耳斯气体的特性函数 $F_m$,并导出其他的热力学函数。

\newpage
\subsection{2-14}
一弹簧在恒温下的恢复力 $F_x$ 与其伸长 $x$ 成正比,即 $F_x = -Ax$,比例系数 $A$ 是温度的函数。今忽略弹簧的热膨胀,试证明弹簧的自由能 $F$,熵 $S$ 和内能 $U$ 的表达式分别为
$$F(T,x) = F(T,0) + \frac{1}{2} Ax^2,$$
$$S(T,x) = S(T,0) - \frac{x^2}{2} \frac{dA}{dT},$$
$$U(T,x) = U(T,0) + \frac{1}{2} \left( A - T \frac{dA}{dT} \right)x^2.$$

\newpage
\subsection{2-15}
X射线衍射实验发现,橡皮带未被拉紧时具有无定形结构;当受张力而被拉伸时,具有晶形结构。这一事实表明,橡皮带具有大的分子链。
(a)试讨论橡皮带在等温过程中被拉伸时,它的熵是增加还是减少;
(b)试证明它的膨胀系数 $\alpha = \frac{1}{L} \left( \frac{\partial L}{\partial T} \right)_\mathcal{F}$ 是负的。

\newpage
\subsection{2-16}
假设太阳是黑体,根据下列数据求太阳表面的温度:单位时间内投射到地球大气层外单位面积上的太阳辐射能量为 $1.35 \times 10^3  \mathrm{J \cdot m^{-2} \cdot s^{-1}}$(该值称为太阳常量),太阳的半径为 $6.955 \times 10^8  \mathrm{m}$,太阳与地球的平均距离为 $1.495 \times 10^{11}  \mathrm{m}$。

\newpage
\subsection{2-17}
计算热辐射在等温过程中体积由 $V_1$ 变到 $V_2$ 时所吸收的热量。

\newpage
\subsection{2-18}
试讨论以平衡辐射为工作物质的卡诺循环,计算其效率。

\newpage
\subsection{2-19}
如图2-3所示,电介质的介电常量 $\varepsilon (T) = \frac{\mathscr{D}}{\mathscr{E}}$ 与温度有关。试求电路为闭路时电介质的热容与充电后再令电路断开后的热容之差。

\newpage
\subsection{2-20}
试证明磁介质 $C_{\mathscr{M}}$ 与 $C_{\mathscr{M}}$ 之差等于
$$C_{\mathscr{H}} - C_{\mathscr{M}} = \mu_0 T \left( \frac{\partial \mathscr{M}}{\partial T} \right)_{\mathscr{H}}^2 \left( \frac{\partial \mathscr{M}}{\partial \mathscr{H}} \right)_T,$$

\newpage
\subsection{2-21}
已知顺磁物质遵从居里定律
$$\mathcal{M} = \frac{C}{T} \mathcal{H} .$$

若维持物质的温度不变,使磁场由0增至 $\mathcal{H}$,求磁化过程释出的热量。

\newpage
\subsection{2-22}
已知超导体的磁感强度 $\mathcal{B} = \mu_0 (\mathcal{H} + \mathcal{M}) = 0$,求证:

(a) $C_{\mathcal{M}}$ 与 $\mathcal{M}$ 无关,只是 $T$ 的函数,其中 $C_{\mathcal{M}}$ 是磁化强度 $\mathcal{M}$ 保持不变时的热容。

(b) $U = \int C_{\mathcal{M}} dT - \frac{\mu_0 \mathcal{M}^2}{2} + U_0$。

(c) $S = \int \frac{C_{\mathcal{M}}}{T} dT + S_0$。

\newpage
\subsection{2-23}
已知顺磁介质遵从居里定律。假设在磁化过程中磁介质的体积变化可以忽略,试分别用 $dW = \mu_0 \mathcal{H} d\mathcal{M}$ 和 $dW = -\mu_0 \mathcal{M} d\mathcal{H}$ 的微功表达式,求单位体积磁介质的自由能、内能和熵,并对所得结果加以解释。