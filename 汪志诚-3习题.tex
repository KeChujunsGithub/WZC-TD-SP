\section{习题3}

\newpage
\subsection{1-3}
简单固体和液体的体胀系数 $\alpha$ 和等温压缩系数 $\kappa_T$ 数值都很小,在一定温度范围内可以把 $\alpha$ 和 $\kappa_T$ 看作常量。试证明简单固体和液体的物态方程可近似表示为 [式 (1.3.14)]

$$ V(T, p) = V_0 (T_0, 0) \left[ 1 + \alpha (T - T_0) - \kappa_T p \right] . $$

\newpage
\subsection{3-1}
证明下列平衡判据(假设 $S>0$):
(a) 在 $S,V$ 不变的情形下,稳定平衡态的 $U$ 最小。
(b) 在 $S,p$ 不变的情形下,稳定平衡态的 $H$ 最小。
(c) 在 $H,p$ 不变的情形下,稳定平衡态的 $S$ 最大。
(d) 在 $F,V$ 不变的情形下,稳定平衡态的 $T$ 最小。
(e) 在 $G,p$ 不变的情形下,稳定平衡态的 $T$ 最小。
(f) 在 $U,S$ 不变的情形下,稳定平衡态的 $V$ 最小。
(g) 在 $F,T$ 不变的情形下,稳定平衡态的 $V$ 最小。

\newpage
\subsection{3-2}
试证明,以内能 $U$ 和体积 $V$ 为自变量、熵的二级微分为
$$ \delta^2 S=\frac{1}{C_vT^2}(\delta U)^2+\frac{2p}{C_vT}\left(\beta-\frac{1}{T}\right)\delta UdV+\left(\frac{2p^2\beta}{C_vT}\frac{p^2}{C_vT^2}\frac{p^2}{C_v}\beta^2-\frac{1}{TVK_T}\right)(\delta V)^2 $$
其中 $\beta=\frac{1}{p}\left(\frac{\partial p}{\partial T}\right)_v$ 是压强系数。

\newpage
\subsection{3-3}
孤立系统含两个子系统。子系统间可以通过做功和传热的方式交换能量。试根据熵判据导出系统达到平衡的平衡条件和平衡稳定条件。

\newpage
\subsection{3-4}
试由 $C_v > 0$ 及 $\left( \frac{\partial p}{\partial V} \right)_T < 0$ 证明 $C_p > 0$ 及 $\left( \frac{\partial p}{\partial V} \right)_S < 0$。

\newpage
\subsection{3-5}
孤立系统含两个子系统。子系统间可以通过做功和传热的方式交换能量。试根据熵判据,从 $\delta^2S<0$ 导出不等式:
$$ \delta p^\alpha \delta V^\alpha - \delta T^\alpha \delta S^\alpha < 0 \quad \alpha = 1, 2 $$
取 $T, V$ 为自变量,可得平衡稳定条件:
$$ C_{v}^\alpha > 0, \left( \frac{\partial V^\alpha}{\partial p} \right)_T < 0, \alpha = 1, 2 $$
取 $S, p$ 为自变量,可得平衡稳定条件:
$$ C_{p}^\alpha > 0, \left( \frac{\partial V^\alpha}{\partial p} \right)_S < 0, \alpha = 1, 2. $$

\newpage
\subsection{3-6}
求证:
(a) $\left( \frac{\partial \mu}{\partial T} \right)_{V,n} = - \left( \frac{\partial S}{\partial n} \right)_{T,V}$;
(b) $\left( \frac{\partial \mu}{\partial p} \right)_{T,n} = \left( \frac{\partial V}{\partial n} \right)_{T,p}$。

\newpage
\subsection{3-7}
求证:
$$ \left( \frac{\partial U}{\partial n} \right)_{T,V} - \mu = -T \left( \frac{\partial \mu}{\partial T} \right)_{V,n} $$

\newpage
\subsection{3-8}
单元两相系与外界隔绝形成孤立系统。试根据熵判据从 $\delta^2 S < 0$ 导出不等式
$$ \delta p^\alpha \delta V^\alpha - \delta T^\alpha \delta S^\alpha < 0, \alpha = 1, 2 $$
取 $T, V$ 为自变量,可得平衡稳定条件
$$ C_v^\alpha > 0, \left( \frac{\partial V^\alpha}{\partial p} \right)_{T} < 0, \alpha = 1, 2 $$
取 $S, P$ 为自变量,可得平衡稳定条件
$$ C_p^\alpha > 0, \left( \frac{\partial V^\alpha}{\partial p} \right)_{S} < 0, \alpha = 1, 2 $$

\newpage
\subsection{3-9}
等温等压下两相共存时,两相系统的定压热容 $C_p = T \left( \frac{\partial S}{\partial T} \right)_p$,体胀系数 $\alpha = \frac{1}{V} \left( \frac{\partial V}{\partial T} \right)_p$ 和等温压缩系数 $\kappa_T = -\frac{1}{V} \left( \frac{\partial V}{\partial p} \right)_T$ 均趋于无穷,试加以说明。

\newpage
\subsection{3-10}
试证明在相变中物质摩尔内能的变化为
$$ \Delta U_m = L \left( 1 - \frac{p}{T} \frac{dT}{dp} \right). $$
如果一相是气相,可看作理想气体,另一相是凝聚相,试将公式化简。

\newpage
\subsection{3-11}
在三相点附近,固态氮的蒸气压力程为
$$ \ln p = 27.92 - \frac{3 \, 754}{T} \quad (\text{SI单位}) $$
液态氮的蒸气压力程为
$$ \ln p = 24.38 - \frac{3 \, 063}{T} \quad (\text{SI单位}) $$
试求氮三相点的温度和压强,氮的汽化热、升华热及在三相点的熔解热。

\newpage
\subsection{3-12}
以 $C_a^\beta$ 表示在维持 $\beta$ 相与 $\alpha$ 相两相平衡的条件下 1 mol $\beta$ 相物质升高 1 K 所吸收的热量,称为 $\beta$ 相的两相平衡摩尔热容,试证明:
$$ C_a^\beta = C_p^\beta - \frac{L}{V_m^\beta - V_m^\alpha} \left( \frac{\partial V_m^\beta}{\partial T} \right)_p. $$
如果 $\beta$ 相是气相,可看作理想气体,$\alpha$ 相是凝聚相,上式可简化为
$$ C_a^\beta = C_p^\beta - \frac{L}{T}, $$
并说明为什么饱和蒸汽的热容有可能是负的。

\newpage
\subsection{3-13}
试证明,相变潜热随温度的变化率为
$$ \frac{dL}{dT} = C_p^\beta - C_p^\alpha + \frac{L}{T} - \left[ \left( \frac{\partial V_m^\beta}{\partial T} \right)_p - \left( \frac{\partial V_m^\alpha}{\partial T} \right)_p \right] \frac{L}{V_m^\beta - V_m^\alpha} $$
如果 $\beta$ 相是气相,$\alpha$ 相是凝聚相,试证明上述式可简化为
$$ \frac{dL}{dT} = C_p^\beta - C_p^\alpha $$

\newpage
\subsection{3-14}
根据式 (3.4.7),利用上题的结果计及潜热 $L$ 是温度的函数,但假设温度的变化范围不大,定压热容可以看作常量,试证明蒸气压力程可以表示为
$$ \ln p = A - \frac{B}{T} + C \ln T. $$

\newpage
\subsection{3-15}
蒸气与液相达到平衡。以 $\frac{dV_m}{dT}$ 表示在维持两相平衡的条件下,蒸气体积随温度的变化率。试证明蒸气的两相平衡膨胀系数为
$$ \frac{1}{V_m} \frac{dV_m}{dT} = \frac{1}{T} \left( 1 - \frac{L}{RT} \right). $$

\newpage
\subsection{3-16}
将范氏气体在不同温度下的等温线的极大点 $N$ 与极小点 $J$ 连起来,可以得到一条曲线 $NCJ$,如图 3-1 所示。试证明这条曲线的方程为
$$ pV_m^3 = a(V_m - 2b), $$
并说明这条曲线划分出来的三个区域 I、II、III 的含义。

\newpage
\subsection{3-17}
证明半径为 $r$ 的肥皂泡的内压与外压之差为 $\frac{4\sigma}{r}$。

\newpage
\subsection{3-18}
证明在曲面分界面的情形下,相变潜热为
$$ L = T(S_m^\beta - S_m^\alpha) = H_m^\beta - H_m^\alpha. $$

\newpage
\subsection{3-19}
证明爱伦费斯特公式:
$$ \frac{dp}{dT} = \frac{\alpha^{(2)} - \alpha^{(1)}}{\kappa_T^{(2)} - \kappa_T^{(1)}}, $$
$$ \frac{dp}{dT} = \frac{C_p^{(2)} - C_p^{(1)}}{TV(\alpha^{(2)} - \alpha^{(1)})}. $$

\newpage
\subsection{3-20}
试根据朗道理论导出单轴铁磁体的熵函数在无序相和有序相的表达式并证明熵函数在临界点是连续的。

\newpage
\subsection{3-21}
承前 2-20 题。假设外磁场十分弱,朗道自由能 (3.9.1) 仍近似适用。试导出无序相和有序相的 $C_{\mathscr{H}} - C_{\mathscr{M}}$。