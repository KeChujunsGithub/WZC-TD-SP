\section{习题5}


\newpage
\subsection{5-1}
带有小孔的隔板将与外界隔绝的容器分为两半。容器中盛有理想气体。两侧气体存在小的温度差 $\Delta T$ 和压强差 $\Delta p$,而各自处在局部平衡。以 $J_n = \frac{dn}{dt}$ 和 $J_u = \frac{dU}{dt}$ 表示单位时间内从左侧转移到右侧的气体的物质的量和内能。试导出气体的熵产生率公式,从而确定相应的动力。

\newpage
\subsection{5-2}
承前 5-1 题,如果流与力之间满足线性关系,即
$$J_u = L_{uu}X_u + L_{un}X_n,$$
$$J_n = L_{nu}X_u + L_{nn}X_n,$$
$$L_{un} = L_{nu} \quad (\text{昂萨格倒易关系}).$$
(a) 试导出 $J_u$ 和 $J_n$ 与温度差 $\Delta T$ 和压强差 $\Delta p$ 的关系。
(b) 证明当 $\Delta T = 0$ 时,由压强差引起的能流和物质流之间满足下述关系:
$$\frac{J_u}{J_n} = \frac{L_{un}}{L_{nn}}.$$
(c) 证明,在没有净物质流通过小孔,即 $J_n = 0$ 时,两侧的压强差与温度差满足
$$\frac{\Delta p}{\Delta T} = \frac{H_m - \frac{L_{un}}{L_{nn}}}{TV_m},$$
其中 $H_m$ 和 $V_m$ 分别是气体的摩尔焓和摩尔体积。以上两式所含 $\frac{L_{un}}{L_{nn}}$ 可由统计物理理论导出(习题 7-14, 7-16)。热力学方法可以把上述两个效应联系起来。

\newpage
\subsection{5-3}
流体含有$k$种化学组元,各组元之间不发生化学反应。系统保持恒温恒压,因而不存在因压强和温度不均匀引起的物质流动和热传导。但存在由于组元浓度在空间分布不均匀引起的扩散。试导出扩散过程的熵流密度和局域熵产生率。

\newpage
\subsection{5-4}
承前5-3题,在粒子流密度与动力呈线性关系的情形下,试就扩散过程证明最小熵产生定理。

\newpage
\subsection{5-5}
系统中存在下述两个化学反应:
$$A+X \xrightarrow{k_1} 2X,$$
$$B+X \xrightarrow{k_3} C.$$
假设反应中不断供给反应物 A 和 B,使其浓度保持恒定,并不断将生成物 C 排除。因此,只有 X 的分子数密度 $n_x$ 可以随时间变化。在扩散可以忽略的情形下,$n_x$ 的变化率为
$$\frac{dn_x}{dt'} = k_1 n_A n_X - k_2 n_X^2 - k_3 n_B n_X.$$
引入变量
$$t = k_2 t',\quad a = \frac{k_1}{k_2} n_A,\quad b = \frac{k_3}{k_2} n_B,\quad X = n_x,$$
上述方程可以表示为
$$\frac{dX}{dt} = (a-b)X - X^2.$$
试求方程的定常解,并分析解的稳定性。

\newpage
\subsection{5-6}
系统中存在下述两个化学反应:
$$A + X \xrightarrow{k_1, k_2} 3X,$$
$$B + X \xrightarrow{k_3} C.$$
假设反应中不断供给反应物 A 和 B,使其浓度保持恒定,并不断将生成物 C 排除,因此只有 X 的浓度 $n_x$ 可以发生改变。假设扩散可以忽略,试写出 $n_x$ 的变化率方程,求方程的定常解,并分析解的稳定性。




